\documentclass[12pt,a4paper]{article}
\usepackage[utf8]{inputenc}
\usepackage[english,russian]{babel}
\usepackage{indentfirst}
\usepackage{misccorr}

\usepackage{graphicx}
\graphicspath{{pictures/}}
\DeclareGraphicsExtensions{.png,.jpg}

\usepackage{amsmath}
\usepackage{hyperref}

\begin{document}
	\begin{titlepage}
		\begin{center}			
			Санкт-Петербургский политехнический университет\\
			Петра Великого
			\vspace{0.25cm}
			
			Институт прикладной математики и механики
			
			Кафедра «Прикладная математика»
			\vfill
			
			\textbf{Отчёт\\
				по лабораторной работе №5\\
				по дисциплине\\
				«Математическая статистика»}\\[5mm]
			\bigskip
		\end{center}
		\vfill
		
		\hfill\begin{minipage}{0.45\textwidth}
			Выполнил студент:
			\vspace{0.2cm}
			
			Кондратьев~Д.~А.\\
			группа: 3630102/70301
		\end{minipage}%
		\bigskip
		
		\hfill\begin{minipage}{0.45\textwidth}
			Проверил:
			\vspace{0.2cm}
			
			к.ф.-м.н., доцент\\
			Баженов Александр Николаевич
		\end{minipage}%
		\vfill
		
		\begin{center}
			Санкт-Петербург\\
			2020 г.
		\end{center}
	\end{titlepage}
	
	
	\tableofcontents{}
	\listoftables
	\listoffigures
	
	\newpage
	\section{Постановка задачи}
	
	Сгенерировать двумерные выборки размерами 20, 60, 100 для нормального двумерного распределения $N(x, y, 0, 0, 1, 1, \rho)$.
	
	Коэффициент корреляции $\rho$ взять равным 0, 0.5, 0.9.
	
	Каждая выборка генерируется 1000 раз и для неё вычисляются: среднее значение, среднее значение квадрата и дисперсия коэффициентов корреляции Пирсона, Спирмена и квадрантного коэффициента корреляции.
	
	Повторить все вычисления для смеси нормальных распределений:
	$$f(x, y) = 0.9N(x, y, 0, 0, 1, 1, 0.9) + 0.1N(x, y, 0, 0, 10, 10, -0.9).$$
	
	Изобразить сгенерированные точки на плоскости и нарисовать эллипс равновероятности.
	
	\section{Теория}
	\subsection{Двумерное нормальное распределение}
	Двумерная случайная величина $(X, Y)$ называется распределённой нормально (или просто нормальной), если её плотность вероятности определена
	формулой:
	\begin{multline}\label{eqn:multivariate_normal}
	N(x, y, \overline{x}, \overline{y}, \sigma_x, \sigma_y, \rho) = \frac{1}{2\pi\sigma_x\sigma_y\sqrt{1 - \rho^2}} \times\\
	\times \exp\left\{
	-\frac{1}{2(1 - \rho^2)} \left[\frac{(x - \overline{x})^2}{\sigma^2_x} - 2\rho\frac{(x - \overline{x})(y - \overline{y})}{\sigma_x \sigma_y} + \frac{(y - \overline{y})^2}{\sigma^2_y}\right]
	\right\}
	\end{multline}
	
	Компоненты $X, Y$ двумерной нормальной случайной величины также распределены нормально с математическими ожиданиями $\overline{x}, \overline{y}$ и средними квадратическими отклонениями $\sigma_x$, $\sigma_y$ соответственно [\ref{Book_1},~с.~133-134].
	
	Параметр $\rho$ называется коэффициентом корреляции.
	
	\subsection{Корреляционный момент(ковариация) и коэффициент корреляции}
	\emph{Корреляционным моментом}, иначе \emph{ковариацией}, двух случайных величин $X$ и $Y$ называется математическое ожидание произведения отклонений этих случайных величин от их математических ожиданий [\ref{Book_1},~с.~141].
	
	\begin{equation}\label{eqn:cor_moment}
	K = cov(X, Y) = M[(X - \overline{x})(Y - \overline{y})]
	\end{equation}
	
	\emph{Коэффициентом корреляции} $\rho$ двух случайных величин X и Y называется отношение их корреляционного момента к произведению их средних квадратических отклонений:
	
	\begin{equation}\label{eqn:cor_coef}
	\rho = \frac{K}{\sigma_x\sigma_y}.
	\end{equation}
	
	\emph{Коэффициент корреляции} --- это нормированная числовая характеристика, являющаяся мерой близости зависимости между случайными величинами
	к линейной [\ref{Book_1},~с.~150].
	
	\subsection{Выборочные коэффициенты корреляции}
	\subsubsection{Выборочный коэффициент корреляции Пирсона}
	Пусть по выборке значений $\{{x_i, y_i}\}^n_1$ двумерной с.в. $(X, Y)$ требуется оценить коэффициент корреляции $\rho$ = $\frac{cov(X, Y)}{\sqrt{DX DY}}$. Естественной оценкой для $\rho$ служит его статистический аналог в виде выборочного коэффициента корреляции, предложенного К.Пирсоном, ---
	\begin{equation}\label{eqn:pears}
	r = \frac{\frac{1}{n} \sum(x_i - \overline{x})(y_i - \overline{y})}{\sqrt{\frac{1}{n} \sum (x_i - \overline{x})^2 \frac{1}{n} \sum(y_i - \overline{y})^2}} = \frac{K}{s_X s_Y}
	\end{equation}
	где $K, s^2_X, s^2_Y$ --- выборочные ковариации и дисперсии с.в. $X$ и $Y$ [\ref{Book_1},~с.~535].
	
	\subsubsection{Выборочный квадрантный коэффициент корреляции}
	Кроме выборочного коэффициента корреляции Пирсона, существуют и другие оценки степени взаимосвязи между случайными величинами. К ним относится \emph{выборочный квадрантный коэффициент корреляции}:
	\begin{equation}\label{eqn:quad}
	r_Q = \frac{(n_1 + n_3) - (n_2 + n_4)}{n}
	\end{equation}
	где $n_1, n_2, n_3 и n_4$ --- количества точке с координатами $(x_i, y_i)$, попавшими
	соответственно в I, II, III и IV квадранты декартовой системы с осями $x' = x - med\:x, y' = y - med\:y$ и с центром в точке с координатами $(med\:x, med\:y)$ [\ref{Book_1},~с.~539].
	
	\subsubsection{Выборочный коэффициент ранговой корреляции Спирмена}
	На практике нередко требуется оценить степень взаимодействия между качественными признаками изучаемого объекта. Качественным называется признак, который нельзя измерить точно, но который позволяет сравнивать изучаемые объекты между собой и располагать их в порядке убывания или возрастания их качества. Для этого объекты выстраиваются в определённом порядке в соответствии с рассматриваемым признаком. Процесс упорядочения называется \emph{ранжированием}, и каждому члену упорядоченной последовательности объектов присваивается ранг, или порядковый номер. Например, объекту с наименьшим значением признака присваивается ранг 1, следующему за ним объекту --- ранг~2, и т.д. Таким образом, происходит сравнение каждого объекта со всеми объектами изучаемой выборки.
	
	Если объект обладает не одним, а двумя качественными признаками --- переменными $X$ и $Y$ , то для исследования их взаимосвязи используют выборочный коэффициент корреляции между двумя последовательностями рангов этих признаков.
	
	Обозначим ранги, соотвествующие значениям переменной $X$, через $u$, а ранги, соотвествующие значениям переменной $Y$, --- через $\upsilon$.
	
	Выборочный коэффициент ранговой корреляции Спирмена определяется как выборочный коэффициент корреляции Пирсона между рангами $u$, $\upsilon$ переменных $X, Y$ :
	\begin{equation}\label{eqn:sperman}
	r_S = \frac{\frac{1}{n} \sum(u_i - \overline{u})(\upsilon_i - \overline{\upsilon})}{\sqrt{\frac{1}{n} \sum (u_i - \overline{u})^2 \frac{1}{n} \sum(\upsilon_i - \overline{\upsilon})^2}},
	\end{equation}
	где $\overline{u} = \overline{\upsilon} = \frac{1 + 2 + .. + n}{n} = \frac{n + 1}{2}$ --- среднее значение рангов [\ref{Book_1},~с.~540-541].
	
	\newpage
	\section{Реализация}	
	\subsection{Выборочные коэффициенты корреляции}
	\begin{center}
		\begin{table}[h!]
			\begin{center}
				\begin{tabular}{|c|c|c|c|}
					\hline
					$\rho=0.0$ (\ref{eqn:cor_coef})& $r$ (\ref{eqn:pears}) & $r_S$ (\ref{eqn:sperman}) & $r_Q$ (\ref{eqn:quad})\\
					\hline
					$E(z)$ & $0.01$ & $0.00$ & $0.00$\\
					\hline
					$E(z^2)$ & $0.05$ & $0.05$ & $0.05$\\
					\hline
					$D(z)$ & $0.0531$ & $0.0520$ & $0.0515$\\
					\hline
					\multicolumn{4}{c}{ } \\
					\hline
					$\rho=0.5$ & $r$ & $r_S$ & $r_Q$\\
					\hline
					$E(z)$ & $0.49$ & $0.46$ & $0.32$ \\
					\hline
					$E(z^2)$ & $0.27$ & $0.25$ & $0.15$ \\
					\hline
					$D(z)$ & $0.0307$ & $0.0348$ & $0.0480$ \\
					\hline
					\multicolumn{4}{c}{ } \\
					\hline
					$\rho=0.9$ & $r$ & $r_S$ & $r_Q$\\
					\hline
					$E(z)$ & $0.896$ & $0.866$ & $0.70$ \\
					\hline
					$E(z^2)$ & $0.805$ & $0.755$ & $0.51$ \\
					\hline
					$D(z)$ & $0.0022$ & $0.0048$ & $0.0291$ \\
					\hline					
				\end{tabular}
				\caption{Двумерное нормальное распределение, $n = 20$}
			\end{center}
		\end{table}
		
		\begin{table}[h!]
			\begin{center}
				\begin{tabular}{|c|c|c|c|}
					\hline
					$\rho=0.0$ & $r$ & $r_S$ & $r_Q$\\
					\hline
					$E(z)$ & $0.00$ & $0.00$ & $-0.00$\\
					\hline
					$E(z^2)$ & $0.02$ & $0.02$ & $0.02$\\
					\hline
					$D(z)$ & $0.0172$ & $0.0175$ & $0.0171$\\
					\hline
					\multicolumn{4}{c}{ } \\
					\hline
					$\rho=0.5$ & $r$ & $r_S$ & $r_Q$\\
					\hline
					$E(z)$ & $0.494$ & $0.47$ & $0.32$\\
					\hline
					$E(z^2)$ & $0.254$ & $0.23$ & $0.12$\\
					\hline
					$D(z)$ & $0.0097$ & $0.0109$ & $0.0147$\\
					\hline
					\multicolumn{4}{c}{ } \\
					\hline
					$\rho=0.9$ & $r$ & $r_S$ & $r_Q$\\
					\hline
					$E(z)$ & $0.8988$ & $0.883$ & $0.706$\\
					\hline
					$E(z^2)$ & $0.8086$ & $0.781$ & $0.507$\\
					\hline
					$D(z)$ & $0.0007$ & $0.0013$ & $0.0089$\\
					\hline					
				\end{tabular}
			\caption{Двумерное нормальное распределение, $n = 60$}
			\end{center}
		\end{table}
		
		\newpage
		\begin{table}[h!]
			\begin{center}
				\begin{tabular}{|c|c|c|c|}
					\hline
					$\rho=0.0$ & $r$ & $r_S$ & $r_Q$\\
					\hline
					$E(z)$ & $-0.001$ & $0.000$ & $-0.00$ \\
					\hline
					$E(z^2)$ & $0.010$ & $0.010$ & $0.01$ \\
					\hline
					$D(z)$ & $0.0098$ & $0.0098$ & $0.0105$ \\
					\hline
					\multicolumn{4}{c}{ } \\
					\hline
					$\rho=0.5$ & $r$ & $r_S$ & $r_Q$\\
					\hline
					$E(z)$ & $0.496$ & $0.476$ & $0.333$ \\
					\hline
					$E(z^2)$ & $0.251$ & $0.233$ & $0.120$ \\
					\hline
					$D(z)$ & $0.0056$ & $0.0064$ & $0.0089$ \\
					\hline
					\multicolumn{4}{c}{ } \\
					\hline
					$\rho=0.9$ & $r$ & $r_S$ & $r_Q$\\
					\hline
					$E(z)$ & $0.8995$ & $0.8868$ & $0.712$ \\
					\hline
					$E(z^2)$ & $0.8094$ & $0.7871$ & $0.511$ \\
					\hline
					$D(z)$ & $0.0004$ & $0.0006$ & $0.0051$ \\
					\hline					
				\end{tabular}
				\caption{Двумерное нормальное распределение, $n = 100$}
			\end{center}
		\end{table}
		
		\begin{table}[h!]
			\begin{center}
				\begin{tabular}{|c|c|c|c|}
					\hline
					$n = 20$ & $r$ & $r_S$ & $r_Q$\\
					\hline
					$E(z)$ & $0.783$ & $0.75$ & $0.56$ \\
					\hline
					$E(z^2)$ & $0.622$ & $0.57$ & $0.35$ \\
					\hline
					$D(z)$ & $0.0085$ & $0.0128$ & $0.0396$ \\
					\hline
					\multicolumn{4}{c}{ } \\
					\hline
					$n = 60$ & $r$ & $r_S$ & $r_Q$\\
					\hline
					$E(z)$ & $0.791$ & $0.768$ & $0.57$ \\
					\hline
					$E(z^2)$ & $0.628$ & $0.594$ & $0.34$ \\
					\hline
					$D(z)$ & $0.0024$ & $0.0033$ & $0.0111$ \\
					\hline
					\multicolumn{4}{c}{ } \\
					\hline
					$n = 100$ & $r$ & $r_S$ & $r_Q$\\
					\hline
					$E(z)$ & $0.789$ & $0.771$ & $0.575$ \\
					\hline
					$E(z^2)$ & $0.624$ & $0.596$ & $0.337$ \\
					\hline
					$D(z)$ & $0.0015$ & $0.0019$ & $0.0063$ \\
					\hline					
				\end{tabular}
			\caption{Смесь нормальных распределений}
			\end{center}
		\end{table}
	\end{center}

	\newpage
	\subsection{Эллипсы рассеивания}
	\begin{center}
		\begin{figure}[h!]
			\includegraphics[width=\textwidth]{"Normal"} 
			\caption[Двумерное нормальное распределение]{Двумерное нормальное распределение}
		\end{figure}
		
		\begin{figure}[h!]
			\includegraphics[width=\textwidth]{"NormalMixed"}
			\caption[Смесь нормальных распределений]{Смесь нормальных распределений}
		\end{figure}
	\end{center}

	\newpage
	\section{Обсуждение}
	Исходя из полученных результатов можно сделать следующие выводы:
	\begin{itemize}
		\item Верны следующие соотношения для дисперсий выборочных коэффициентов корреляции:
		\begin{itemize}
			\item для двумерного нормального распределения: $r < r_S < r_Q$;
			
			\item для смеси нормальных распределений: $r < r_S < r_Q$.
		\end{itemize}
	
		\item Процент попавших элементов выборки в эллипс рассеивания (95\%-ная доверительная область) примерно равен его теоретическому значению (95\%).
		
		\item При уменьшении корреляции эллипс равновероятности стремится к окружности, а при увеличении --- растягивается.
	\end{itemize}
	
	\section{Литература}
	\begin{enumerate}
		\item \label{Book_1} \textbf{Вероятностные разделы математики.} Учебник для бакалавров технических направлений.//Под ред. Максимова~Ю.Д. --- Спб.: «Иван Федоров», 2001. --- 592 c., илл.
		
		\item Correlation and dependence. URL: \url{https://en.wikipedia.org/wiki/Correlation_and_dependence}
	\end{enumerate}

	\section{Приложение}
	\begin{enumerate}
		\item Код лабораторной. URL: \url{https://github.com/DmitriiKondratev/MatStat/blob/master/Lab_5/Lab_5.ipynb}
		
		\item Код отчёта. URL: \url{https://github.com/DmitriiKondratev/MatStat/blob/master/Lab_5/Lab_report_5.tex}
		
	\end{enumerate}
\end{document}