\documentclass[12pt,a4paper]{article}
\usepackage[utf8]{inputenc}
\usepackage[english,russian]{babel}
\usepackage{indentfirst}
\usepackage{misccorr}

\usepackage{amsmath}
\usepackage{hyperref}

\begin{document}
	\begin{titlepage}
		\begin{center}			
			Санкт-Петербургский политехнический университет\\
			Петра Великого
			\vspace{0.25cm}
			
			Институт прикладной математики и механики
			
			Кафедра «Прикладная математика»
			\vfill
			
			\textbf{Отчёт\\
				по лабораторной работе №2\\
				по дисциплине\\
				«Математическая статистика»}\\[5mm]
			\bigskip
		\end{center}
		\vfill
		
		\hfill\begin{minipage}{0.45\textwidth}
			Выполнил студент:
			\vspace{0.2cm}
			
			Кондратьев~Д.~А.\\
			группа: 3630102/70301
		\end{minipage}%
		\bigskip
		
		\hfill\begin{minipage}{0.45\textwidth}
			Проверил:
			\vspace{0.2cm}
			
			к.ф.-м.н., доцент\\
			Баженов Александр Николаевич
		\end{minipage}%
		\vfill
		
		\begin{center}
			Санкт-Петербург\\
			2020 г.
		\end{center}
	\end{titlepage}
	
	
	\tableofcontents{}
	\listoftables
	\newpage
	
	\section{Постановка задачи}
	
	Для 5-ти рапределений:
	\begin{itemize}
		\item Нормальное распределение $N(x,0,1)$;
		\item Распределение Коши $C(x,0,1)$;
		\item Распределение Лапласа $L( x,0,\frac{1}{\sqrt{2}})$;
		\item Распределение Пуассона $P(k, 10)$;
		\item Равномерное Распределение $U(x,-\sqrt{3}, \sqrt{3})$;
	\end{itemize}
	Сгенерировать выборки размером 10, 100 и 1000 элементов. Для каждой выборки вычислить следующие статистические характеристики положения данных: $\overline{x}, med\;x, z_R, z_Q, z_{tr}$. Повторить такие вычисления 1000 раз для каждой выборки и найти среднее характеристик положения и их квадратов:
	\begin{equation}\label{eqn:expected_value}
	E(z)=\overline{z}
	\end{equation}
	Вычислить оценку дисперсии по формуле:
	\begin{equation}\label{eqn:variance}
	D(z)=\overline{z^2}-\overline{z}^2
	\end{equation}
	Представить полученные данные в виде таблиц.
	
	\section{Теория}
	\subsection{Распределения}
	\begin{itemize}
		\item Нормальное распределение \begin{equation}\label{eqn:normal}
		N(x,0,1) = \frac{1}{\sqrt{2\pi}}e^{-\frac{x^2}{2}}
		\end{equation}
		
		\item Распределение Коши
		\begin{equation}\label{eqn:cauchy}
		C(x,0,1) = \frac{1}{\pi(1+x^2)}
		\end{equation}
		
		\item Распределение Лапласа
		\begin{equation}\label{eqn:laplace}
		L\left( x,0,\frac{1}{\sqrt{2}}\right) = \frac{1}{\sqrt{2}}e^{-\sqrt{2}\vert x\vert}
		\end{equation}
		
		\item Распределение Пуассона
		\begin{equation}\label{eqn:poisson}
		P(k,10) = \frac{10^k}{k!}e^{-10}
		\end{equation}
		
		\item Равномерное Распределение
		\begin{equation}\label{eqn:uniform}
		U(x,-\sqrt{3}, \sqrt{3}) = 
		\begin{cases}
		\frac{1}{2\sqrt{3}} &\vert x\vert \leqslant \sqrt{3}\\
		0 &\vert x\vert > \sqrt{3}
		\end{cases}
		\end{equation}
	\end{itemize}
	
	\subsection{Вариационный ряд}
		Вариационным рядом называется последовательность элементов выборки, расположенных в неубывающем порядке. Одинаковые элементы повторяются  [\ref{Book_1}, с. 409].
		
		Запись вариационного ряда: $x_1, x_2, ... , x_n$.
		
		Элементы вариационного ряда $x_i\ (i = 1,2, ... ,n)$ называются порядковыми статистиками.
	
	\subsection{Выборочные числовые характеристики}
		С помощью выборки образуются её числовые характеристики. Это числовые характеристики дискретной случайной величины $X^*$, принимающей выборочные значения $x_1, x_2, ... , x_n$ [\ref{Book_1}, с. 411].
		
	\subsubsection{Характеристики положения}
	\begin{itemize}
		\item Выборочное среднее:
		\begin{equation}\label{eqn:average}
			\overline{x} = \frac{1}{n}\sum_{i=1}^n x_i
		\end{equation}
		
		\item Выборочная медиана:
		\begin{equation}\label{eqn:med}
			med\;x =
			\begin{cases}
				x_{k+1}, & n = 2k+1\\
				\frac{1}{2}\left(x_k+x_{k+1}\right), & n = 2k
			\end{cases}
		\end{equation}
		
		\item Полусумма экстремальных выборочных элементов:
		\begin{equation}\label{eqn:mean_extr}
			z_R = \frac{1}{2}\left(x_1+x_n\right)
		\end{equation}
		
		\item Полусумма квартилей:
		\begin{equation}\label{eqn:quartiles}
			z_Q = \frac{1}{2}\left(z_{\frac{1}{4}}+z_{\frac{3}{4}}\right) 
		\end{equation}
		
		\item Усечённое среднее:
		\begin{equation}\label{eqn:trim_mean}
			z_{tr} = \frac{1}{n - 2r}\sum_{i=r+1}^{n-r} x_i,\ r\approx\frac{n}{4}
		\end{equation}
	\end{itemize}

	\subsubsection{Характеристики рассеяния}
		Выборочная дисперсия:
		\begin{equation}\label{eqn:var}
			D = \frac{1}{n}\sum_{i=1}^n (x_i-\overline{x})^2
		\end{equation}
	
	\section{Реализация}
	Лабораторная работа выполнена на программном языке \emph{Python\;3.8} в среде разработки \emph{Jupyter Notebook\;6.0.3}. В работе использовались следующие пакеты языка \emph{Python}:
	\begin{itemize}
		\item \emph{numpy} --- для генерации выборки и работы с массивами;
		\item \emph{scipy.stats} --- содержит все необходимые распределения.
	\end{itemize}
	Ссылка на исходный код лабораторной работы приведена в приложении.

	\newpage
	\section{Результаты}
	\subsection{Характеристики положения и рассеяния}
	\begin{center}
		\begin{table}[h]
			\caption{Нормальное распределение}
			\begin{center}
				\begin{tabular}{|c|c|c|c|c|c|c|}
					\hline
					& \multicolumn{2}{c|}{$n=10$} & \multicolumn{2}{c|}{$n=100$} & \multicolumn{2}{c|}{$n=1000$}\\
					\hline
					 & $E(x)$ (\ref{eqn:expected_value}) & $D(x)$ (\ref{eqn:variance}) & $E(x)$ & $D(x)$ & $E(x)$ & $D(x)$\\
					\hline
					$\overline{x}$ (\ref{eqn:average}) & $0.00$ & $0.0985$ & $0.00$ & $0.0107$ & $-0.002$ & $0.0010$\\
					\hline
					$med\;x$ (\ref{eqn:med}) & $0.0$ & $0.1337$ & $0.00$ & $0.0149$ & $-0.001$ & $0.0016$\\
					\hline
					$z_R$ (\ref{eqn:mean_extr}) & $0.0$ & $0.1847$ & $0.00$ & $0.0928$ & $-0.01$ & $0.0630$\\
					\hline
					$z_Q$ (\ref{eqn:quartiles}) & $0.0$ & $0.1177$ & $-0.01$ & $0.0132$ & $-0.004$ & $0.0012$\\
					\hline
					$z_{tr} (\ref{eqn:trim_mean}) $ & $0.0$ & $0.1113$ & $0.00$ & $0.0123$ & $-0.002$ & $0.0012$\\
					\hline
				\end{tabular}
			\end{center}
		\end{table}
	
		\begin{table}[h]
			\caption{Распределение Коши}
			\begin{center}
				\begin{tabular}{|c|c|c|c|c|c|c|}
					\hline
					& \multicolumn{2}{c|}{$n=10$} & \multicolumn{2}{c|}{$n=100$} & \multicolumn{2}{c|}{$n=1000$}\\
					\hline
					& $E(x)$ & $D(x)$ & $E(x)$ & $D(x)$ & $E(x)$ & $D(x)$\\
					\hline
					$\overline{x}$ & $2.9$ & $8261.4971$ & $-0.3$ & $282.5623$ & $-0.7$ & $1680.6291$\\
					\hline
					$med\;x$ & $0.00$ & $0.3821$ & $-0.01$ & $0.0260$ & $-0.002$ & $0.0024$\\
					\hline
					$z_R$ & $14.2$ & $206063$ & $-10.5$ & $693687$ & $-358.5$ & $417271337$\\
					\hline
					$z_Q$ & $0.0$ & $1.2177$ & $-0.03$ & $0.0493$ & $-0.003$ & $0.0049$\\
					\hline
					$z_{tr}$ & $0.0$ & $0.5952$ & $0.00$ & $0.0261$ & $-0.001$ & $0.0026$\\
					\hline
				\end{tabular}
			\end{center}
		\end{table}
	
		\begin{table}[h!]
			\caption{Распределение Лапласа}
			\begin{center}
				\begin{tabular}{|c|c|c|c|c|c|c|}
					\hline
					& \multicolumn{2}{c|}{$n=10$} & \multicolumn{2}{c|}{$n=100$} & \multicolumn{2}{c|}{$n=1000$}\\
					\hline
					& $E(x)$ & $D(x)$ & $E(x)$ & $D(x)$ & $E(x)$ & $D(x)$\\
					\hline
					$\overline{x}$ & $-0.02$ & $0.0897$ & $-0.002$ & $0.0099$ & $0.002$ & $0.001$\\
					\hline
					$med\;x$ & $-0.02$ & $0.0648$ & $0.003$ & $0.0058$ & $0.0003$ & $0.0005$\\
					\hline
					$z_R$ & $0.0$ & $0.3887$ & $0.0$ & $0.409$ & $0.0$ & $0.4084$\\
					\hline
					$z_Q$ & $-0.01$ & $0.0943$ & $-0.01$ & $0.01$ & $-0.001$ & $0.001$\\
					\hline
					$z_{tr}$ & $-0.01$ & $0.0662$ & $0.001$ & $0.0063$ & $0.0007$ & $0.0006$\\
					\hline
				\end{tabular}
			\end{center}
		\end{table}
	
		\begin{table}[h]
			\caption{Распределение Пуассона}
			\begin{center}
				\begin{tabular}{|c|c|c|c|c|c|c|}
					\hline
					& \multicolumn{2}{c|}{$n=10$} & \multicolumn{2}{c|}{$n=100$} & \multicolumn{2}{c|}{$n=1000$}\\
					\hline
					& $E(x)$ & $D(x)$ & $E(x)$ & $D(x)$ & $E(x)$ & $D(x)$\\
					\hline
					$\overline{x}$ & $10.02$ & $1.0132$ & $10.00$ & $0.0976$ & $9.994$ & $0.0098$\\
					\hline
					$med\;x$ & $9.9$ & $1.4617$ & $9.9$ & $0.1944$ & $9.995$ & $0.005$\\
					\hline
					$z_R$ & $10.4$ & $1.9290$ & $10.95$ & $1.0299$ & $11.8$ & $0.6673$\\
					\hline
					$z_Q$ & $9.9$ & $1.2086$ & $9.9$ & $0.1590$ & $9.995$ & $0.0037$\\
					\hline
					$z_{tr}$ & $9.9$ & $1.1459$ & $9.9$ & $0.1166$ & $9.85$ & $0.0107$\\
					\hline
				\end{tabular}
			\end{center}
		\end{table}
	
		\begin{table}[h]
			\caption{Равномерное распределение}
			\begin{center}
				\begin{tabular}{|c|c|c|c|c|c|c|}
					\hline
					& \multicolumn{2}{c|}{$n=10$} & \multicolumn{2}{c|}{$n=100$} & \multicolumn{2}{c|}{$n=1000$}\\
					\hline
					& $E(x)$ & $D(x)$ & $E(x)$ & $D(x)$ & $E(x)$ & $D(x)$\\
					\hline
					$\overline{x}$ & $-0.02$ & $0.0893$ & $0.000$ & $0.0099$ & $0.001$ & $0.001$\\
					\hline
					$med\;x$ & $0.0$ & $0.2088$ & $0.01$ & $0.0294$ & $0.001$ & $0.003$\\
					\hline
					$z_R$ & $-0.01$ & $0.0392$ & $0.0002$ & $0.0006$ & $0.0001$ & $0.0000$\\
					\hline
					$z_Q$ & $0.0$ & $0.1282$ & $-0.02$ & $0.0146$ & $-0.001$ & $0.0014$\\
					\hline
					$z_{tr}$ & $0.0$ & $0.1479$ & $0.00$ & $0.0200$ & $0.001$ & $0.0019$\\
					\hline
				\end{tabular}
			\end{center}
		\end{table}
	\end{center}

	\section{Обсуждение}
	Исходя из полученных результатов можно сделать следующие выводы:
	\begin{itemize}
		\item Дисперсия может гарантировать порядок точности среднего значения только до первой значащей цифры после запятой в дисперсии включительно. Поэтому есть необходимость проверки данных результатов на совпадение этой точности.
		
		\item При увеличении количества выборки гарантируемая точность будет возрастать.
		
		\item Предыдущие выводы касаются всех распределений, кроме распределения Коши. Так как оно имеет бесконечную дисперсию и, следовательно, никакой точности гарантировать не может.
		
		\newpage
		\item Можно вывести следующее отношение для характеристик положения при $n=1000$:
		\begin{enumerate}
			\item для нормального распределения: $z_R < z_Q < z_{tr} \leq \overline{x} < med\;x$;
			
			\item для распределения Коши: $z_R < \overline{x} < z_Q < med\;x < z_{tr}$;
			
			\item для распределения Лапласа: $z_Q < z_R < med\;x < z_{tr} < \overline{x}$;
			
			\item для распределения Пуассона: $ z_{tr} < \overline{x} < z_Q \leq med\;x < z_R$;
			
			\item для равномерного распределения: $z_Q < z_R < z_{tr} \leq med\;x \leq \overline{x}$.
		\end{enumerate}
	\end{itemize}
	
	\section{Литература}
	\begin{enumerate}
		\item \label{Book_1} \textbf{Вероятностные разделы математики.} Учебник для бакалавров технических направлений.//Под ред. Максимова~Ю.Д. --- Спб.: «Иван Федоров», 2001. --- 592 c., илл.
	\end{enumerate}

	\section{Приложение}
	\begin{enumerate}
		\item Код лабораторной. URL: \url{https://github.com/DmitriiKondratev/MatStat/blob/master/Lab_2/Lab_2.ipynb}
		
		\item Код отчёта. URL: \url{https://github.com/DmitriiKondratev/MatStat/blob/master/Lab_2/Lab_report_2.tex}
		
	\end{enumerate}
\end{document}