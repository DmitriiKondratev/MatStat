\documentclass[12pt,a4paper]{article}
\usepackage[utf8]{inputenc}
\usepackage[english,russian]{babel}
\usepackage{indentfirst}
\usepackage{misccorr}

\usepackage{graphicx}
\graphicspath{{pictures/}}
\DeclareGraphicsExtensions{.png,.jpg}

\usepackage{amsmath}
\usepackage{hyperref}

\begin{document}
	\begin{titlepage}
		\begin{center}			
			Санкт-Петербургский политехнический университет\\
			Петра Великого
			\vspace{0.25cm}
			
			Институт прикладной математики и механики
			
			Кафедра «Прикладная математика»
			\vfill
			
			\textbf{Отчёт\\
				по лабораторной работе №4\\
				по дисциплине\\
				«Математическая статистика»}\\[5mm]
			\bigskip
		\end{center}
		\vfill
		
		\hfill\begin{minipage}{0.45\textwidth}
			Выполнил студент:
			\vspace{0.2cm}
			
			Кондратьев~Д.~А.\\
			группа: 3630102/70301
		\end{minipage}%
		\bigskip
		
		\hfill\begin{minipage}{0.45\textwidth}
			Проверил:
			\vspace{0.2cm}
			
			к.ф.-м.н., доцент\\
			Баженов Александр Николаевич
		\end{minipage}%
		\vfill
		
		\begin{center}
			Санкт-Петербург\\
			2020 г.
		\end{center}
	\end{titlepage}
	
	
	
	\tableofcontents{}
	\listoffigures
	
	\newpage
	\section{Постановка задачи}
	
	Для 5-ти рапределений:
	\begin{itemize}
		\item Нормальное распределение $N(x,0,1)$;
		\item Распределение Коши $C(x,0,1)$;
		\item Распределение Лапласа $L( x,0,\frac{1}{\sqrt{2}})$;
		\item Распределение Пуассона $P(k, 10)$;
		\item Равномерное Распределение $U(x,-\sqrt{3}, \sqrt{3})$;
	\end{itemize}
	Сгенерировать выборки размером 20, 60 и 100 элементов. Построить на них эмпирические функции распределения и ядерные оценки плотности распределения на отрезке $[-4;4]$ для непрерывных распределений и на отрезке $[6;14]$ для распределения Пуассона.
	
	\section{Теория}
	\subsection{Распределения}
	\begin{itemize}
		\item Нормальное распределение \begin{equation}\label{eqn:normal}
		N(x,0,1) = \frac{1}{\sqrt{2\pi}}e^{-\frac{x^2}{2}}
		\end{equation}
		
		\item Распределение Коши
		\begin{equation}\label{eqn:cauchy}
		C(x,0,1) = \frac{1}{\pi(1+x^2)}
		\end{equation}
		
		\item Распределение Лапласа
		\begin{equation}\label{eqn:laplace}
		L\left( x,0,\frac{1}{\sqrt{2}}\right) = \frac{1}{\sqrt{2}}e^{-\sqrt{2}\vert x\vert}
		\end{equation}
		
		\item Распределение Пуассона
		\begin{equation}\label{eqn:poisson}
		P(k,10) = \frac{10^k}{k!}e^{-10}
		\end{equation}
		
		\item Равномерное Распределение
		\begin{equation}\label{eqn:uniform}
		U(x,-\sqrt{3}, \sqrt{3}) = 
		\begin{cases}
		\frac{1}{2\sqrt{3}} &\vert x\vert \leqslant \sqrt{3}\\
		0 &\vert x\vert > \sqrt{3}
		\end{cases}
		\end{equation}
	\end{itemize}
	
	\subsection{Эмпирическая функция распределения}
	\subsubsection{Статистический ряд}
		Статистическим рядом называется последовательность различных элементов выборки $z_1,z_2, ...,z_k$, расположенных в возрастающем порядке с указанием частот $n_1,n_2, ...,n_k$, с которыми эти элементы содержатся в выборке. Статистический ряд обычно записывается в виде таблицы:
		\begin{center}
			\begin{table}[h]
				\begin{center}
					\begin{tabular}{|c|c|c|c|c|}
						\hline
						$z$ & $z_1$ & $z_2$ & ... & $z_k$\\
						\hline
						$n$ & $n_1$ & $n_2$ & ... & $n_k$\\
						\hline
					\end{tabular}
				\end{center}
			\caption{Статистический ряд}
			\end{table}
		\end{center}
	
	\subsubsection{Определение}
		Эмпирической (выборочной) функцией распределения (э.~ф.~р.) называется относительная частота события $X < x$, полученная по данной выборке:
		\begin{equation}\label{eqn:edf_prob}
			F^*_n(x) = P^*(X < x)
		\end{equation}
		
	\subsubsection{Описание}
		Для получения относительной частоты $P^*(X < x)$ просуммируем в статистическом ряде, построенном по данной выборке, все частоты $n_i$, для которых элементы $z_i$ статистического ряда меньше $x$.\\ Тогда $P^*(X<x) = \frac{1}{n}\sum_{z_i < x}n_i$. Получаем
		\begin{equation}\label{eqn:edf_sum}
			F^*(x) = \frac{1}{n}\sum_{z_i < x}n_i
		\end{equation}
		$F^*(x)$ ---  функция распределения дискретной случайной величины X*, заданной таблицей распределения:
		\begin{center}
			\begin{table}[h]
				\begin{center}
					\begin{tabular}{|c|c|c|c|c|}
						\hline
						$z$ & $z_1$ & $z_2$ & ... & $z_k$\\
						\hline
						$n$ & $n_1$ & $n_2$ & ... & $n_k$\\
						\hline
					\end{tabular}
				\end{center}
				\caption{Таблица распределения}
			\end{table}
		\end{center}
		Эмпирическая функция распределения является оценкой, т. е. приближённым значением, генеральной функции распределения:
		\begin{equation}\label{eqn:edf_approx}
		F^*_n(x) \approx F_X(x).
		\end{equation}
	
	\subsection{Оценки плотности вероятности}
	\subsubsection{Определение}
		Оценкой плотности вероятности $f(x)$ называется функция $\hat{f}(x)$, построенная на основе выборки, приближённо равная $f(x)$:
		\begin{equation}\label{eqn:pdf_approx}
			\hat{f}(x) \approx f(x).
		\end{equation}
	\subsubsection{Ядерные оценки}
		Представим оценку в виде суммы с числом слагаемых, равным объёму выборки:
		\begin{equation}\label{eqn:pdf_sum}
			\hat{f}(x) = \frac{1}{nh_n}\sum_{i=1}^{n}K(\frac{x-x_i}{h_n}).
		\end{equation}
		Здесь функция $K(u)$, называемая ядерной (ядром), непрерывна и является плотностью вероятности, $x_1, ... ,x_n$ --- элементы выборки, ${h_n}$ --- любая последовательность положительных чисел, обладающая свойствами:
		\begin{equation}\label{eqn:h cond}
			h_n \xrightarrow[n\rightarrow\infty]{}0; \quad \frac{h_n}{n^{-1}}\xrightarrow[n\rightarrow\infty]{}\infty.
		\end{equation}
		Такие оценки называются непрерывными ядерными [\ref{Book_1}, с. 421-423].
		
		Замечание. Свойство, означающее сближение оценки с оцениваемой величиной при $n\rightarrow\infty$ в каком-либо смысле, называется состоятельностью оценки. 
		
		Если плотность $f(x)$ кусочно-непрерывная, то ядерная оценка плотности является состоятельной при соблюдении условий, накладываемых на параметр сглаживания $h_n$, а также на ядро $K(u)$.
		
		Гауссово (нормальное) ядро [\ref{Book_2}, с. 38]:
		\begin{equation}\label{eqn:Kernel fun}
			K(u)=\frac{1}{\sqrt{2\pi}}e^{-\frac{u^2}{2}}.
		\end{equation}
		
		Правило Сильвермана [\ref{Book_2}, с. 44]:
		\begin{equation}\label{eqn:Silverman rule}
		h_n = 1.06\hat{\sigma}n^{-1/5},
		\end{equation}
		где $\hat{\sigma}$ --- выборочное стандартное отклонение.
	\section{Реализация}
		Лабораторная работа выполнена на программном языке \emph{Python\;3.8} в среде разработки \emph{Jupyter Notebook\;6.0.3}. В работе использовались следующие пакеты языка \emph{Python}:
		\begin{itemize}
			\item \emph{numpy} --- для генерации выборки и работы с массивами;
			
			\item \emph{matplotlib.pyplot} и \emph{seaborn} --- для построения графиков;
			
			\item \emph{scipy.stats} --- содержит все необходимые распределения.
		\end{itemize}
		Ссылка на исходный код лабораторной работы приведена в приложении.

	\section{Результаты}
	\subsection{Эмпирическая функция распределения}
	\begin{center}
		\begin{figure}[h!]
			\includegraphics[width=\textwidth]{"Normal distribution (edf)"} 
			\caption[Нормальное распределение]{Нормальное распределение}
		\end{figure}
		
		\begin{figure}[h!]
			\includegraphics[width=\textwidth]{"Cauchy distribution (edf)"}
			\caption[Распределение Коши]{Распределение Коши}
		\end{figure}
	
		\begin{figure}[h!]
			\includegraphics[width=\textwidth]{"Laplace distribution (edf)"}
			\caption[Распределение Лапласа]{Распределение Лапласа}
		\end{figure}
		
		\begin{figure}[h!]
			\includegraphics[width=\textwidth]{"Poisson distribution (edf)"}
			\caption[Распределение Пуассона]{Распределение Пуассона}
		\end{figure}
	
		\begin{figure}[h!]			
			\includegraphics[width=\textwidth]{"Uniform distribution (edf)"}
			\caption[Равномерное распределение]{Равномерное распределение}
		\end{figure}
	\end{center}
	
	\newpage
	\subsection{Ядерные оценки плотности распределения}
	\begin{center}
		\begin{figure}[hp!]
			\includegraphics[width=\textwidth]{"Normal distribution (kernel)"} 
			\caption[Нормальное распределение]{Нормальное распределение}
		\end{figure}
		
		\begin{figure}[hp!]
			\includegraphics[width=\textwidth]{"Cauchy distribution (kernel)"}
			\caption[Распределение Коши]{Распределение Коши}
		\end{figure}
		
		\begin{figure}[hp!]
			\includegraphics[width=\textwidth]{"Laplace distribution (kernel)"}
			\caption[Распределение Лапласа]{Распределение Лапласа}
		\end{figure}
		
		\begin{figure}[hp!]
			\includegraphics[width=\textwidth]{"Poisson distribution (kernel)"}
			\caption[Распределение Пуассона]{Распределение Пуассона}
		\end{figure}
		
		\begin{figure}[hp!]			
			\includegraphics[width=\textwidth]{"Uniform distribution (kernel)"}
			\caption[Равномерное распределение]{Равномерное распределение}
		\end{figure}
	\end{center}

	\newpage
	\section{Обсуждение}
	Исходя из полученных результатов можно сделать следующие выводы:
	\begin{itemize}
		\item При увеличении размера выборки  качество оценки эмперической функцией эталонной функции распределения возрастает.
		
		\item При увеличении размера выборки  качество ядерной оценки плотности распределения возрастает.
		
		\item Наилучшее качество ядерной оценки плотности распределения достигается при следующих параметрах:
		\begin{enumerate}
			\item для нормального распределения --- $h=h_n$;
			
			\item для распределения Коши --- $h=h_n$;
			
			\item для распределения Лапласа --- $h=\frac{h_n}{2}$ или $h=h_n$;
			
			\item для распределения Пуассона --- $h=2h_n$;
			
			\item для равномерного распределения --- $h=\frac{h_n}{2}$ или $h=h_n$.
		\end{enumerate}
	\end{itemize}
	
	\section{Литература}
	\begin{enumerate}
		\item \label{Book_1} \textbf{Вероятностные разделы математики.} Учебник для бакалавров технических направлений.//Под ред. Максимова~Ю.Д. --- Спб.: «Иван Федоров», 2001. --- 592 c., илл.
		
		\item \label{Book_2} Анатольев, Станислав (2009) «Непараметрическая регрессия», Квантиль, №7, стр. 37-52.
		
		\item Kernel density estimation. URL: \url{https://en.wikipedia.org/wiki/Kernel_density_estimation}
	\end{enumerate}

	\section{Приложение}
	\begin{enumerate}
		\item Код лабораторной. URL: \url{https://github.com/DmitriiKondratev/MatStat/blob/master/Lab_4/Lab_4.ipynb}
		
		\item Код отчёта. URL: \url{https://github.com/DmitriiKondratev/MatStat/blob/master/Lab_4/Lab_report_4.tex}
		
	\end{enumerate}
\end{document}