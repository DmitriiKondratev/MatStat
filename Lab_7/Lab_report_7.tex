\documentclass[12pt,a4paper]{article}
\usepackage[utf8]{inputenc}
\usepackage[english,russian]{babel}
\usepackage{indentfirst}
\usepackage{misccorr}

\usepackage{graphicx}
\graphicspath{{pictures/}}
\DeclareGraphicsExtensions{.png,.jpg}

\usepackage{float}
\usepackage{amsmath}
\usepackage{hyperref}

\begin{document}
	\begin{titlepage}
		\begin{center}			
			Санкт-Петербургский политехнический университет\\
			Петра Великого
			\vspace{0.25cm}
			
			Институт прикладной математики и механики
			
			Кафедра «Прикладная математика»
			\vfill
			
			\textbf{Отчёт\\
				по лабораторной работе №7\\
				по дисциплине\\
				«Математическая статистика»}\\[5mm]
			\bigskip
		\end{center}
		\vfill
		
		\hfill\begin{minipage}{0.45\textwidth}
			Выполнил студент:
			\vspace{0.2cm}
			
			Кондратьев~Д.~А.\\
			группа: 3630102/70301
		\end{minipage}%
		\bigskip
		
		\hfill\begin{minipage}{0.45\textwidth}
			Проверил:
			\vspace{0.2cm}
			
			к.ф.-м.н., доцент\\
			Баженов Александр Николаевич
		\end{minipage}%
		\vfill
		
		\begin{center}
			Санкт-Петербург\\
			2020 г.
		\end{center}
	\end{titlepage}
	
	
	
	\tableofcontents{}
	\listoftables
	
	\newpage
	\section{Постановка задачи}
	Сгенерировать выборку объёмом 100 элементов для нормального распределения $N(x, 0, 1)$. По сгенерированной выборке оценить параметры $\mu$ и $\sigma$ нормального закона методом максимального правдоподобия. В качестве основной гипотезы $H_0$ будем считать, что сгенерированное распределение имеет вид $N(x, \hat{\mu}, \hat{\sigma})$. Проверить основную гипотезу, используя критерий согласия $\chi^2$. В качестве уровня значимости взять $\alpha = 0.05$. Привести таблицу вычислений $\chi^2$.
	\section{Теория}
	\subsection{Метод максимального правдоподобия}
	Метод максимального правдоподобия --- метод оценивания неизвестного параметра путём максимимзации функции правдоподобия.
	
	\begin{equation}
	\hat{\theta}_{\text{МП}} = \arg \max_\theta L(x_1,x_2,\ldots,x_n,\theta)
	\end{equation}
	где $L$ --- функция правдоподобия, которая представляет собой совместную плотность вероятности независимых случайных величин $x_1,x_2,\ldots,x_n$ и является функцией неизвестного параметра $\theta$.
	\begin{equation}
	L = f(x_1,\theta)\cdot f(x_2,\theta)\cdot\cdots\cdot f(x_n,\theta)
	\end{equation}
	Оценкой максимального правдоподобия будем называть такое значение $\hat{\theta}_{\text{МП}}$ из множества допустимых значений параметра $\theta,$ для которого функция правдоподобия принимает при заданных $x_1, x_2, \ldots, x_n$ максимальное значение.
	
	Тогда при оценивании математического ожидания $m$ и дисперсии $\sigma^2$ нормального распределения $N(m,\sigma)$ получим:
	\begin{equation}
	\ln(L)=-\frac{n}{2}\ln(2\pi)-\frac{n}{2}\ln\left(\sigma^2\right)-\frac{1}{2\sigma^2}\sum\limits_{i=1}^n(x_i-m)^2
	\end{equation}
	
	\subsection{Проверка гипотезы о законе распределения генеральной совокупности. Метод хи-квадрат}
	Разобьём генеральную совокупность на $k$ неперсекающихся подмножеств $\Delta_1, \Delta_2,\ldots, \Delta_k,\;\Delta_i = (a_i,a_{i+1}],$ $p_i = P(X\in\Delta_i),\;i=1,2,\ldots,k\;$ --- вероятность того, что точка попала в $i$-ый промежуток.
	
	Так как генеральная совокупность --- это $\mathbb{R}$, то крайние промежутки будут бесконечными: $\Delta_1 = (-\infty,a_1], \;\Delta_k = (a_k,\infty), \;p_i = F(a_i) - F(a_{i-1})$.
	
	$n_i\;\--$ частота попадания выборочных элементов в $\Delta_i,\;i=1,2,\ldots,k.$
	
	В случае справедливости гипотезы $H_0$ относительно частоты $\frac{n_i}{n}$ при больших $n$ должны быть близки к $p_i,$ значит в качестве меры имеет смысл взять: 
	
	\begin{equation}
	Z = \sum\limits_{i=1}^k\frac{n}{p_i}\left(\frac{n_i}{n}-p_i\right)^2
	\end{equation}
	
	Тогда:
	\begin{equation}
	\chi^2_B=\sum\limits_{i=1}^k\frac{n}{p_i}\left(\frac{n_i}{n}-p_i\right)^2=\sum\limits_{i=1}^k\frac{(n_i-np_i)^2}{np_i}
	\end{equation}
	
	\textbf{Теорема К.Пирсона.} Статистика критерия $\chi^2$ асимптотически распределена по закону $\chi^2$ с $k - 1$ степенями свободы.
	
	\section{Реализация}
		Лабораторная работа выполнена на программном языке \emph{Python\;3.8} в среде разработки \emph{Jupyter Notebook\;6.0.3}. В работе использовались следующие пакеты языка \emph{Python}:
		\begin{itemize}
			\item \emph{numpy} --- для генерации выборки и работы с массивами;
			
			\item \emph{tabulate} --- для построения таблиц;
			
			\item \emph{scipy.stats} --- содержит необходимые распределения и критерий $\chi^2$.
		\end{itemize}
		Ссылка на исходный код лабораторной работы приведена в приложении.

	\section{Результаты}
		\subsection{Проверка гипотезы о законе распределения генеральной совокупности. Метод хи-квадрат}
		Метод максимального правдоподобия:
		$$\hat{\mu} \approx 0.01, \quad \hat{\sigma} \approx 1.02.$$
		
		Критерий согласия $\chi^2$:\\
		Количество промежутков $k = 6$.\\
		Уровень значимости $\alpha = 0.05$.
		\begin{table}[H]
			\begin{center}
				\begin{tabular}{|c|c|c|c|c|c|c|}
					\hline
					$i$ & $\Delta_i$          &   $n_i$ &   $p_i$ &   $np_i$ &   $n_i-np_i$ &   $\frac{(n_i-np_i)^2}{np_i}$ \\
					\hline
					   1   & $(-\infty, -1.01]$ & $16$  & $0.1562$ & $15.62$ & $0.38$  & $0.01$ \\
					   2   & $(-1.01, -0.37]$   & $17$  & $0.2004$ & $20.04$ & $-3.04$ & $0.46$ \\
					   3   & $(-0.37, 0.28]$    & $31$  & $0.2517$ & $25.17$ & $5.83$  & $1.35$ \\
					   4   & $(0.28, 0.92]$     & $18$  & $0.2122$ & $21.22$ & $-3.22$ & $0.49$ \\
					   5   & $(0.92, 1.56]$     & $10$  & $0.1201$ & $12.01$ & $-2.01$ & $0.34$ \\
					   6   & $(1.56, \infty]$   & $8$   & $0.0594$ & $5.94$  & $2.06$  & $0.72$ \\
					$\sum$ & ---                & $100$ & $1.0000$ & $100$   & $0.00$  & $3.36 = \chi^2_B$ \\
					\hline
				\end{tabular}
			\end{center}
			\caption{Вычисление $\chi^2_B$ при проверке гипотизы $H_0$ о нормальном законе распределения N($x, \hat{\mu}, \hat{\sigma}$)}
		\end{table}

	\section{Обсуждение}
	Табличное значение квартиля $\chi^2_{1 - \alpha}(k - 1) = \chi^2_{0.95}(5) = 11.07$.
	
	Исходя из полученных результатов можно сделать следующие выводы:
	\begin{itemize}
		\item Так как $\chi^2_B < \chi^2_{0.95}(5)$, то можно заключить, что гипотеза $H_0$ о нормальном законе распределения $N(x, \hat{\mu}, \hat{\sigma})$ на уровне значимости $\alpha = 0.05$, согласуется с выборкой.
	\end{itemize}
	
	\section{Литература}
	\begin{enumerate}		
		\item Maximum likelihood estimation. URL: \url{https://en.wikipedia.org/wiki/Maximum_likelihood_estimation}
		
		\item Chi-squared test. URL: \url{https://en.wikipedia.org/wiki/Chi-squared_test}
	\end{enumerate}

	\section{Приложение}
	\begin{enumerate}
		\item Код лабораторной. URL: \url{https://github.com/DmitriiKondratev/MatStat/blob/master/Lab_7/Lab_7.ipynb}
		
		\item Код отчёта. URL: \url{https://github.com/DmitriiKondratev/MatStat/blob/master/Lab_7/Lab_report_7.tex}
		
	\end{enumerate}
\end{document}