\documentclass[12pt,a4paper]{article}
\usepackage[utf8]{inputenc}
\usepackage[english,russian]{babel}
\usepackage{indentfirst}
\usepackage{misccorr}

\usepackage{graphicx}
\DeclareGraphicsExtensions{.png,.jpg}

\usepackage{float}
\usepackage{amsmath}
\usepackage{hyperref}

\begin{document}
	\begin{titlepage}
		\begin{center}			
			Санкт-Петербургский политехнический университет\\
			Петра Великого
			\vspace{0.25cm}
			
			Институт прикладной математики и механики
			
			Высшая школа прикладной математики и вычислительной физики
			\vfill
			
			\textbf{Отчёт\\
				по курсовой работе\\
				по дисциплине\\
				«Математическая статистика»}\\[5mm]
			\bigskip
		\end{center}
		\vfill
		
		\hfill\begin{minipage}{0.45\textwidth}
			Выполнил студент:
			\vspace{0.2cm}
			
			Кондратьев~Д.~А.\\
			группа: 3630102/70301
		\end{minipage}%
		\bigskip
		
		\hfill\begin{minipage}{0.45\textwidth}
			Проверил:
			\vspace{0.2cm}
			
			к.ф.-м.н., доцент\\
			Баженов Александр Николаевич
		\end{minipage}%
		\vfill
		
		\begin{center}
			Санкт-Петербург\\
			2020 г.
		\end{center}
	\end{titlepage}
	
	
	
	\tableofcontents{}
	\listoftables
	\listoffigures
	
	\newpage
	\section{Постановка задачи}
		Есть набор 2D данных в текстовом формате – следы жизни в геологических
		объектах. Образцы взяты с двух разных регионов:
		\begin{itemize}
			\item русского севера;
			\item центральной Африки.
		\end{itemize}
	
		На объект подавалось излучение от ближнего ультрафиолетового до видимого. Длина волны — первая переменная $x_1$.
		
		Когда свет с заданной $x_1$ попадал в объект, его поглощали молекулы и в свою очередь, излучали свет с длинами волны $x_2$ примерно в том же диапазоне.
		
		То, что они излучали записывается в виде графика $I(x_1 = const, x_2)$.
		
		Далее, $x_1$ варьируются, и формируется $I(x_1, x_2
		)$. Функция 2-х переменных.
		
		Пики на графике $I$ можно идентифицировать с излучением протеиногенных аминокислот, т.е. это остатки органической жизни.
		
		Известна область для каждой аминокислоты в координатах $(x_1, x_2)$.
		
		\hfill
		
		Для классификации двух типов данных ранее был предложен параметр K (на основании C, A, B, T), который позволяет достаточно уверенно проводить разделение этих типов. При этом не использовались данные по переменной M.
		
		Необходимо:
		\begin{itemize}
			\item построить двумерное поле $(M, K)$ для Африки и Арктики;
			\item проанализировать полученный результат.
		\end{itemize}
	
	\section{Теория}
		\subsection {Используемые параметры}
		\begin{table}[H]
			\begin{center}
				\begin{tabular}{|c|p{100pt}|c|c|}
					\hline
					Буквенное обозначение & Тип компонента & $E_{x_{max}}(nm)$ & $E_{m_{max}}(nm)$ \\
					\hline
					C & Humic-like & $320-350$ & $420-480$ \\
					\hline
					A & Humic-like & $250-260$ & $380-480$ \\
					\hline
					M & Mariane
					Humic-like & $310-320$ & $380-420$ \\
					\hline
					B & Tysone-like,
					Protein-like & $270-280$ & $300-320$ \\
					\hline
					T & Tryptophane-like,
					Protein-like or
					phenol-like & $270-280$ & $320-350$ \\
					\hline
				\end{tabular}
				\caption{Таблица интенсивностей}
				\label{table:intensity}
			\end{center}
		\end{table}
		
		Параметр $K$, ранее использовавшийся для сравнения данных, выражет отношение сложной и простой органики и вычисляется по формуле:
		\begin{equation}\label{eqn:K}
			K = \frac{C+A}{B+T}
		\end{equation}
	
		\subsection {Подготовка данных}
		По полученным данным были получены изображения, далее были обрезаны релеевские облучения и выделены области в соответствии с таблицей интенсивностей [\ref{table:intensity}]. После были посчитаны суммарные интенсивности каждых областей, по которым в дальнейшем будет вестить исследование.
		\begin{figure}[H]
			\includegraphics[width=\textwidth]{"../Course_work/pictures/example"} 
			\caption[Обработанный файл с выделенными областями]{Обработанный файл с выделенными областями}
		\end{figure}
	
	\section{Реализация}
		Курсовая работа выполнена на программном языке \emph{Python\;3.8} в среде разработки \emph{Jupyter Notebook\;6.0.3}. В работе использовались следующие пакеты языка \emph{Python}:
		\begin{itemize}
			\item \emph{numpy} --- для обработки исходных данных и работы с массивами;
			
			\item \emph{matplotlib} --- для визуализации результатов.
		\end{itemize}
		Ссылка на исходный код курсовой работы приведена в приложении.
		
	\newpage
	\section{Результаты}
		Исходя из полученных данных $(M, K)$ были построены следующие двумерные поля:
		\begin{figure}[H]
			\includegraphics[width=\textwidth]{"../Course_work/pictures/fields"} 
			\caption[Двумерные поля Африки и севера России]{Двумерные поля Африки и севера России}
		\end{figure}
	
		\begin{figure}[H]
			\includegraphics[width=\textwidth]{"../Course_work/pictures/common_picture"} 
			\caption[Совместное двумерное поле Африки и севера России]{Совместное двумерное поле Африки и севера России}
		\end{figure}
	
	\section{Обсуждение}
	Исходя из полученных результатов можно сделать следующие выводы:
	\begin{itemize}
		\item Для севера России характерен больший разброс точек по сравнению с Африкой, где они расположены более кучно.
		
		\item Для почти всех координат севера России справедливо утверждение, что при увеличении $M$ $K$ возрастает.
		
		\item Для определения области лучше опираться на параметр K, так как он дает более четкую картину поведения аминокислот, характерную для конкретную область.
		
		\item Также, используя данные поля, можно поставить задачу классификации и построить алгоритм, определяющий область по координатам $(M,K)$.
	\end{itemize}
	
	\section{Литература}
	\begin{enumerate}
		\item Документация \emph{numpy}. URL: \url{https://numpy.org/doc/stable/reference/}
		\item Документация \emph{matplotlib}. URL: \url{https://matplotlib.org/3.2.1/contents.html}
	\end{enumerate}

	\section{Приложение}
	\begin{enumerate}
		\item Код лабораторной. URL: \url{https://github.com/DmitriiKondratev/MatStat/blob/master/Course_work/Course_work.ipynb}
		
		\item Код отчёта. URL: \url{https://github.com/DmitriiKondratev/MatStat/blob/master/Course_work/Course_work_report.tex}
		
	\end{enumerate}
\end{document}