\documentclass[12pt,a4paper]{article}
\usepackage[utf8]{inputenc}
\usepackage[english,russian]{babel}
\usepackage{indentfirst}
\usepackage{misccorr}

\usepackage{graphicx}
\DeclareGraphicsExtensions{.png,.jpg}

\usepackage{float}
\usepackage{amsmath}
\usepackage{hyperref}

\begin{document}
	\begin{titlepage}
		\begin{center}			
			Санкт-Петербургский политехнический университет\\
			Петра Великого
			\vspace{0.25cm}
			
			Институт прикладной математики и механики
			
			Кафедра «Прикладная математика»
			\vfill
			
			\textbf{Отчёт\\
				по лабораторной работе №8\\
				по дисциплине\\
				«Математическая статистика»}\\[5mm]
			\bigskip
		\end{center}
		\vfill
		
		\hfill\begin{minipage}{0.45\textwidth}
			Выполнил студент:
			\vspace{0.2cm}
			
			Кондратьев~Д.~А.\\
			группа: 3630102/70301
		\end{minipage}%
		\bigskip
		
		\hfill\begin{minipage}{0.45\textwidth}
			Проверил:
			\vspace{0.2cm}
			
			к.ф.-м.н., доцент\\
			Баженов Александр Николаевич
		\end{minipage}%
		\vfill
		
		\begin{center}
			Санкт-Петербург\\
			2020 г.
		\end{center}
	\end{titlepage}
	
	
	
	\tableofcontents{}
	\listoftables
	
	\newpage
	\section{Постановка задачи}
		Для двух выборок размерами 20 и 100 элементов, сгенерированных согласно нормальному закону $N(x, 0, 1)$, для параметров положения и масштаба построить асимптотически нормальные интервальные оценки на основе точечных оценок метода максимального правдоподобия и классические интервальные оценки на основе статистик $\chi^2$ и Стьюдента. В качестве параметра надёжности взять $\gamma = 0.95$.
	
	\section{Теория}
	\subsection{Доверительные интервалы для параметров нормального распределения}
	Оценкой максимального правдоподобия для математического ожидания  является среднее арифметическое: $\mu=\frac{1}{n}\sum\limits_{i=1}^nx_i.$
	
	Оценка максимального правдоподобия для дисперсии вычисляется по формуле: $\sigma^2 = \frac{1}{n}\sum\limits_{i=1}^n(x_i-\overline{x})^2.$
	
	Доверительным интервалом или интервальной оценкой числовой характеристики или параметра распределения $\theta$ с доверительной вероятностью $\gamma$ называется интервал со случайными границами $(\theta_1,\theta_2),$ содержащий параметр $\theta$ с вероятностью $\gamma$.
	
	Функция распределения Стьюдента:
	\begin{equation}
	T = \sqrt{n-1}\frac{\overline{x}-\mu}{\delta}
	\end{equation}
	
	Функция плотности распределения $\chi^2$:
	\begin{equation}
	f(x) =
	\begin{cases}
	0,&x\leq 0\\
	\frac{1}{2^\frac{n}{2}\Gamma\left(\frac{n}{2}\right)}x^{\frac{n}{2}-1}e^{-\frac{x}{2}},& x>0
	\end{cases}
	\end{equation}

	Интервальная оценка математического ожидания:
	\begin{equation}
	P=\left(\overline{x}-\frac{\sigma t_{1-\frac{\alpha}{2}}(n-1)}{\sqrt{n-1}}<\mu<\overline{x}+\frac{\sigma t_{1-\frac{\alpha}{2}}(n-1)}{\sqrt{n-1}}\right) = \gamma,
	\end{equation}
	где $t_{1-\frac{\alpha}{2}}$ --- квантиль распределения Стьюдента порядка $1-\frac{\alpha}{2}$.
	
	Интервальная оценка дисперсии:
	\begin{equation}
	P=\left(\frac{\sigma\sqrt{n}}{\sqrt{\chi^2_{1-\frac{\alpha}{2}}(n-1)}}<\sigma<\frac{\sigma\sqrt{n}}{\sqrt{\chi^2_\frac{\alpha}{2}(n-1)}}\right) = \gamma,
	\end{equation}
	где $\chi_{1-\frac{\alpha}{2}}^2,\;\chi_\frac{\alpha}{2}^2\;$ --- квантили распределения Стьюдента порядков $1-\frac{\alpha}{2}$ и $\frac{\alpha}{2}$ соответственно.
	
	\subsection{Доверительные интервалы для математического ожидания m и среднего квадратического отклонения $\sigma$ произвольного распределения при большом объёме выборки. Асимптотический подход}
	
	Асимптотическая интервальная оценка математического ожидания:
	\begin{equation}
	P = \left(\overline{x}-\frac{\sigma u_{1-\frac{\alpha}{2}}}{\sqrt{n}}<m<\overline{x}+\frac{\sigma u_{1-\frac{\alpha}{2}}}{\sqrt{n}}\right)=\gamma,
	\end{equation}
	где $u_{1-\frac{\alpha}{2}}\;\--$ квантиль нормального распределения $N(x,0,1)$ порядка $1-\frac{\alpha}{2}$.
	\begin{equation}
	\sigma(1 - 0.5u_{1 - \alpha/2} \sqrt{e + 2}/ \sqrt{n}) < \sigma < \sigma(1 + 0.5u_{1 - \alpha/2} \sqrt{e + 2}/ \sqrt{n})
	\end{equation}
	
	\section{Реализация}
		Лабораторная работа выполнена на программном языке \emph{Python\;3.8} в среде разработки \emph{Jupyter Notebook\;6.0.3}. В работе использовались следующие пакеты языка \emph{Python}:
		\begin{itemize}
			\item \emph{numpy} --- для генерации выборки и работы с массивами;
			
			\item \emph{scipy.stats} --- содержит все необходимые распределения.
		\end{itemize}
		Ссылка на исходный код лабораторной работы приведена в приложении.

	\section{Результаты}
		\subsection{Доверительные интервалы для параметров нормального распределения}
		\begin{table}[H]
			\begin{center}
				\begin{tabular}{|c|c|c|}
					\hline
					$n$ & $m$ & $\sigma$\\
					\hline
					$20$ & $(-0.62; 0.28)$ & $(0.73; 1.40)$\\ 
					\hline
					$100$ & $(-0.24; 0.12)$ & $(0.81; 1.07)$\\
					\hline
				\end{tabular}
			\end{center}
			\caption{Доверительные интервалы для параметров нормального распределения}
		\end{table}
		
		\subsection{Доверительные интервалы для параметров произвольного распределения. Асимптотический подход}
		\begin{table}[H]
			\begin{center}				
				\begin{tabular}{|c|c|c|}
					\hline
					$n$ & $m$ & $\sigma$\\
					\hline
					$20$ & $(-0.58; 0.24)$ & $(0.83; 1.09)$\\ 
					\hline
					$100$ & $(-0.24; 0.12)$ & $(0.86; 1.01)$\\
					\hline
				\end{tabular}
			\end{center}
			\caption{Доверительные интервалы для параметров произвольного распределения. Асимптотический подход}
		\end{table}
	
	\section{Обсуждение}
	Исходя из полученных результатов можно сделать следующие выводы:
	\begin{itemize}
		\item Генеральные характеристики ($m = 0$ и $\sigma = 1$) накрываются построенными доверительными интервалами.
		
		\item Лучший результат достигается на выборках большого объема, так как получаемые интервалы получаются меньшей длины.
		
		\item Доверительные интервалы для параметров нормального распределения более надёжны, так как основаны на точном, а не асимптотическом распределении.
	\end{itemize}
	
	\section{Литература}
	\begin{enumerate}
		\item \label{Book_1} \textbf{Вероятностные разделы математики.} Учебник для бакалавров технических направлений.//Под ред. Максимова~Ю.Д. --- Спб.: «Иван Федоров», 2001. --- 592 c., илл.
		
		\item Confidence interval. URL: \url{https://en.wikipedia.org/wiki/Confidence_interval}
	\end{enumerate}

	\section{Приложение}
	\begin{enumerate}
		\item Код лабораторной. URL: \url{https://github.com/DmitriiKondratev/MatStat/blob/master/Lab_8/Lab_8.ipynb}
		
		\item Код отчёта. URL: \url{https://github.com/DmitriiKondratev/MatStat/blob/master/Lab_8/Lab_report_8.tex}
		
	\end{enumerate}
\end{document}