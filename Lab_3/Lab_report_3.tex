\documentclass[12pt,a4paper]{article}
\usepackage[utf8]{inputenc}
\usepackage[english,russian]{babel}
\usepackage{indentfirst}
\usepackage{misccorr}

\usepackage{graphicx}
\graphicspath{{pictures/}}
\DeclareGraphicsExtensions{.png,.jpg}

\usepackage{amsmath}
\usepackage{hyperref}

\begin{document}
	\begin{titlepage}
		\begin{center}			
			Санкт-Петербургский политехнический университет\\
			Петра Великого
			\vspace{0.25cm}
			
			Институт прикладной математики и механики
			
			Кафедра «Прикладная математика»
			\vfill
			
			\textbf{Отчёт\\
				по лабораторной работе №3\\
				по дисциплине\\
				«Математическая статистика»}\\[5mm]
			\bigskip
		\end{center}
		\vfill
		
		\hfill\begin{minipage}{0.45\textwidth}
			Выполнил студент:
			\vspace{0.2cm}
			
			Кондратьев~Д.~А.\\
			группа: 3630102/70301
		\end{minipage}%
		\bigskip
		
		\hfill\begin{minipage}{0.45\textwidth}
			Проверил:
			\vspace{0.2cm}
			
			к.ф.-м.н., доцент\\
			Баженов Александр Николаевич
		\end{minipage}%
		\vfill
		
		\begin{center}
			Санкт-Петербург\\
			2020 г.
		\end{center}
	\end{titlepage}
	
	
	
	\tableofcontents{}
	\listoftables
	\listoffigures
	
	\newpage
	\section{Постановка задачи}
	
	Для 5-ти рапределений:
	\begin{itemize}
		\item Нормальное распределение $N(x,0,1)$;
		\item Распределение Коши $C(x,0,1)$;
		\item Распределение Лапласа $L( x,0,\frac{1}{\sqrt{2}})$;
		\item Распределение Пуассона $P(k, 10)$;
		\item Равномерное Распределение $U(x,-\sqrt{3}, \sqrt{3})$;
	\end{itemize}
	Сгенерировать выборки размером 20 и 100 элементов. Построить для них боксплот Тьюки.
	
	Для каждого распределения определить долю выбросов экспериментально (сгенерировав выборку, соответствующую распределению 1000 раз, и вычислив среднюю долю выбросов и их дисперсии) и сравнить с результатами, полученными теоретически.\\
	Средняя доля выбросов:
	\begin{equation}\label{eqn:expected_value}
		E(z)=\overline{z}
	\end{equation}
	Дисперсия:
	\begin{equation}\label{eqn:variance}
		D(z)=\overline{z^2}-\overline{z}^2
	\end{equation}
	
	\section{Теория}
	\subsection{Распределения}
	\begin{itemize}
		\item Нормальное распределение \begin{equation}\label{eqn:normal}
		N(x,0,1) = \frac{1}{\sqrt{2\pi}}e^{-\frac{x^2}{2}}
		\end{equation}
		
		\item Распределение Коши
		\begin{equation}\label{eqn:cauchy}
		C(x,0,1) = \frac{1}{\pi(1+x^2)}
		\end{equation}
		
		\item Распределение Лапласа
		\begin{equation}\label{eqn:laplace}
		L\left( x,0,\frac{1}{\sqrt{2}}\right) = \frac{1}{\sqrt{2}}e^{-\sqrt{2}\vert x\vert}
		\end{equation}
		
		\item Распределение Пуассона
		\begin{equation}\label{eqn:poisson}
		P(k,10) = \frac{10^k}{k!}e^{-10}
		\end{equation}
		
		\item Равномерное Распределение
		\begin{equation}\label{eqn:uniform}
		U(x,-\sqrt{3}, \sqrt{3}) = 
		\begin{cases}
		\frac{1}{2\sqrt{3}} &\vert x\vert \leqslant \sqrt{3}\\
		0 &\vert x\vert > \sqrt{3}
		\end{cases}
		\end{equation}
	\end{itemize}
	
	\subsection{Боксплот Тьюки}
	\subsubsection{Определение}
		Боксплот (англ. box plot) — график, использующийся в описательной статистике, компактно изображающий одномерное распределение вероятностей.
		
	\subsubsection{Описание}
		Такой вид диаграммы в удобной форме показывает медиану, нижний и верхний квартили и выбросы. Несколько таких ящиков можно нарисовать бок о бок, чтобы визуально сравнивать одно распределение с другим; их можно располагать как горизонтально, так и вертикально. Расстояния между различными частями ящика позволяют определить степень разброса (дисперсии) и асимметрии данных и выявить выбросы [\ref{Box plot}].
		
	\subsubsection{Построение}
		Границами ящика служат первый и третий квартили, линия в середине
		ящика — медиана. Концы усов — края статистически значимой выборки
		(без выбросов). Длину «усов» определяют разность первого квартиля и полутора межквартильных расстояний и сумма третьего квартиля и полутора
		межквартильных расстояний. Формула имеет вид:
		\begin{equation}\label{eqn:box_plot}
		X_1 = Q_1 - \frac{3}{2}(Q_3 - Q_1),\  X_2 = Q_3 + \frac{3}{2}(Q_3 - Q_1)
		\end{equation}
		где $X_1$ --- нижняя граница уса, $X_2$ --- верхняя граица уса, $Q_1$ --- первый квартиль, $Q_3$ --- третий квартиль.
		
		Данные, выходящие за границы усов (выбросы), отображаются на графике
		в виде маленьких кружков.
		
		
		
	\subsection{Теоретическая вероятность выбросов}
		Встроенными средствами языка программирования \emph{Python} можно вычислить теоретические первый и третий квартили распределений ($Q^T_1$ и $Q^T_3$ соответственно). По формуле (\ref{eqn:box_plot}) можно вычислить теоретические нижнюю и верхнюю границы уса ($X^T_1$ и $X^T_3$ соответственно). Выбросами считаются величины $x$, такие что:
		\begin{equation}\label{eqn:outliers}
		\left[ \begin{array}{l}
			x < X^T_1 \\
			x > X^T_3
		\end{array} \right.
		\end{equation}
		Теоретическая вероятность выбросов для непрерывных распределений:
		\begin{equation}\label{eqn:continous_prob}
			P^T_{out} = P(x < X^T_1) + P(x > X^T_2) = F(X^T_1) + (1-F(X^T_2)),
		\end{equation}
		где $F(X) = P(x \leq X)$ --- функция распределения.
		Теоретическая вероятность выбросов для дискретных распределений:
		\begin{equation}\label{eqn:discrete_prob}
			P^T_{out} = P(x < X^T_1) + P(x > X^T_2) = (F(X^T_1) - P(x = X^T_1)) + (1-F(X^T_2)),
		\end{equation}
		где $F(X) = P(x \leq X)$ --- функция распределения.
		
	\section{Реализация}
	Лабораторная работа выполнена на программном языке \emph{Python\;3.8} в среде разработки \emph{Jupyter Notebook\;6.0.3}. В работе использовались следующие пакеты языка \emph{Python}:
	\begin{itemize}
		\item \emph{numpy} --- для генерации выборки и работы с массивами;
		
		\item \emph{matplotlib.pyplot} --- для построения боксплотов Тьюки;
		
		\item \emph{scipy.stats} --- содержит все необходимые распределения, а также именно с помощью него можно получить теоретические оценки.
	\end{itemize}
	Ссылка на исходный код лабораторной работы приведена в приложении.

	\section{Результаты}
	\subsection{Боксплот Тьюки}
	\begin{center}
		\begin{figure}[hp!]
			\includegraphics[width=\textwidth]{"Normal distribution"} 
			\caption[Нормальное распределение]{Нормальное распределение}
			
			\includegraphics[width=\textwidth]{"Cauchy distribution"}
			\caption[Распределение Коши]{Распределение Коши}
		\end{figure}
		
		\begin{figure}[hp!]
			\includegraphics[width=\textwidth]{"Laplace distribution"}
			\caption[Распределение Лапласа]{Распределение Лапласа}
			
			\includegraphics[width=\textwidth]{"Poisson distribution"}
			\caption[Распределение Пуассона]{Распределение Пуассона}
		\end{figure}
	
		\begin{figure}[h!]
			\includegraphics[width=\textwidth]{"Uniform distribution"}
			\caption[Равномерное распределение]{Равномерное распределение}
		\end{figure}
	\end{center}
	
	\newpage
	\subsection{Доля выбросов}
	\begin{center}
		\begin{table}[h]
			\caption{Доля выбросов}
			\begin{center}
				\begin{tabular}{|c|c|c|c|c|}
					\hline
					& \multicolumn{2}{c|}{$n=20$} & \multicolumn{2}{c|}{$n=100$}\\
					\cline{2-5}
					\raisebox{1.5ex}[0cm][0cm]{Распределение}
					& $E$ (\ref{eqn:expected_value}) & $D$ (\ref{eqn:variance}) & $E$ & $D$\\
					\hline
					Нормальное распределение & $0.024$ & $0.0019$ & $0.0101$ & $0.0002$ \\
					\hline
					Распределение Коши & $0.152$ & $0.0049$ & $0.154$ & $0.0011$\\
					\hline
					Распределение Лапласа & $0.074$ & $0.0044$ & $0.0632$ & $0.0009$\\
					\hline
					Распределение Пуассона & $0.024$ & $0.0022$ & $0.0105$ & $0.0002$\\
					\hline
					Равномерное распределение & $0.0020$ & $0.0002$ & $0.0$ & $0.0$\\
					\hline
				\end{tabular}
			\end{center}
		\end{table}
	\end{center}
	
	\newpage
	\subsection{Теоретическая вероятность выбросов}
	\begin{center}
		\begin{table}[h]
			\caption{Теоретическая вероятность выбросов}
			\begin{center}
				\begin{tabular}{|c|c|c|c|c|c|}
					\hline
					Распределение & $Q^T_1$ & $Q^T_3$ & $X^T_1$ (\ref{eqn:box_plot}) & $X^T_2$ (\ref{eqn:box_plot}) & $P^T_{out}$ (\ref{eqn:continous_prob}), (\ref{eqn:discrete_prob})\\
					\hline
					Нормальное распределение & $-0.6745$ & $0.6745$ & $-2.698$ & $2.698$ & $0.007$\\
					\hline
					Распределение Коши & $-1$ & $1$ & $-4$ & $4$ & $0.156$\\
					\hline
					Распределение Лапласа & $-0.4901$ & $0.4901$ & $-1.9605$ & $1.9605$ & $0.0625$\\
					\hline
					Распределение Пуассона & $8$ & $12$ & $2$ & $18$ & $0.0077$\\
					\hline
					Равномерное распределение & $-0.866$ & $0.866$ & $-3.4641$ & $3.4641$ & $0$\\
					\hline
				\end{tabular}
			\end{center}
		\end{table}
	\end{center}

	\section{Обсуждение}
	Исходя из полученных результатов можно сделать следующие выводы:
	\begin{itemize}
		\item Справедливо следующее соотношение для долей выбросов:\\
		равномерное распределение < нормальное распределение < распределение Пуассона < распределение Лапласа < распределение Коши.
		
		\item Доля выбросов, полученная экспериментально, близка с результатами, полученными теоретически.
	\end{itemize}
	
	\section{Литература}
	\begin{enumerate}
		\item \label{Box plot} Box plot. URL: \url{https://en.wikipedia.org/wiki/Box_plot}
	\end{enumerate}

	\section{Приложение}
	\begin{enumerate}
		\item Код лабораторной. URL: \url{https://github.com/DmitriiKondratev/MatStat/blob/master/Lab_3/Lab_3.ipynb}
		
		\item Код отчёта. URL: \url{https://github.com/DmitriiKondratev/MatStat/blob/master/Lab_3/Lab_report_3.tex}
		
	\end{enumerate}
\end{document}