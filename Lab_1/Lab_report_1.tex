\documentclass[12pt,a4paper]{article}
\usepackage[utf8]{inputenc}
\usepackage[english,russian]{babel}
\usepackage{indentfirst}
\usepackage{misccorr}

\usepackage{graphicx}
\graphicspath{{pictures/}}
\DeclareGraphicsExtensions{.png,.jpg}

\usepackage{amsmath}
\usepackage{hyperref}

\begin{document}
	\begin{titlepage}
		\begin{center}			
			Санкт-Петербургский политехнический университет\\
			Петра Великого
			\vspace{0.25cm}
			
			Институт прикладной математики и механики
			
			Кафедра «Прикладная математика»
			\vfill
			
			\textbf{Отчёт\\
				по лабораторной работе №1\\
				по дисциплине\\
				«Математическая статистика»}\\[5mm]
			\bigskip
		\end{center}
		\vfill
		
		\hfill\begin{minipage}{0.45\textwidth}
			Выполнил студент:
			\vspace{0.2cm}
			
			Кондратьев~Д.~А.\\
			группа: 3630102/70301
		\end{minipage}%
		\bigskip
		
		\hfill\begin{minipage}{0.45\textwidth}
			Проверил:
			\vspace{0.2cm}
			
			к.ф.-м.н., доцент\\
			Баженов Александр Николаевич
		\end{minipage}%
		\vfill
		
		\begin{center}
			Санкт-Петербург\\
			2020 г.
		\end{center}
	\end{titlepage}
	
	
	
\tableofcontents{}
\newpage

\section{Постановка задачи}

	Для 5-ти рапределений:
	\begin{itemize}
		\item Нормальное распределение $N(x,0,1)$;
		\item Распределение Коши $C(x,0,1)$;
		\item Распределение Лапласа $L( x,0,\frac{1}{\sqrt{2}})$;
		\item Распределение Пуассона $P(k, 10)$;
		\item Равномерное Распределение $U(x,-\sqrt{3}, \sqrt{3})$;
	\end{itemize}
	Сгенерировать выборки размером 10, 50 и 1000 элементов.
	Построить на одном рисунке гистограмму и график плотности распределения.

\section{Теория}
	\subsection{Распределения}
	\begin{itemize}
		\item Нормальное распределение \begin{equation}\label{eqn:normal}
		N(x,0,1) = \frac{1}{\sqrt{2\pi}}e^{-\frac{x^2}{2}}
		\end{equation}
		
		\item Распределение Коши
		\begin{equation}\label{eqn:cauchy}
		C(x,0,1) = \frac{1}{\pi(1+x^2)}
		\end{equation}
		
		\item Распределение Лапласа
		\begin{equation}\label{eqn:laplace}
		L\left( x,0,\frac{1}{\sqrt{2}}\right) = \frac{1}{\sqrt{2}}e^{-\sqrt{2}\vert x\vert}
		\end{equation}
		
		\item Распределение Пуассона
		\begin{equation}\label{eqn:poisson}
		P(k,10) = \frac{10^k}{k!}e^{-10}
		\end{equation}
		
		\item Равномерное Распределение
		\begin{equation}\label{eqn:uniform}
		U(x,-\sqrt{3}, \sqrt{3}) = 
		\begin{cases}
		\frac{1}{2\sqrt{3}} &\vert x\vert \leqslant \sqrt{3}\\
		0 &\vert x\vert > \sqrt{3}
		\end{cases}
		\end{equation}
	\end{itemize}

	\subsection{Гистограмма}
		\subsubsection{Определение}
			Гистограмма в математической статистике --- это функция, приближающая
			плотность вероятности некоторого распределения, построенная на основе
			выборки из него [\ref{Histogram}].

		\subsubsection{Графическое описание}
			Графически гистограмма строится следующим образом. Сначала множество значений, которое может принимать элемент выборки, разбивается на
			несколько интервалов. Чаще всего эти интервалы берут одинаковыми, но
			это не является строгим требованием. Эти интервалы откладываются на
			горизонтальной оси, затем над каждым рисуется прямоугольник. Если все
			интервалы были одинаковыми, то высота каждого прямоугольника пропорциональна числу элементов выборки, попадающих в соответствующий
			интервал. Если интервалы разные, то высота прямоугольника выбирается
			таким образом, чтобы его площадь была пропорциональна числу элементов
			выборки, которые попали в этот интервал [\ref{Histogram}].

		\subsubsection{Использование}
			Гистограммы применяются в основном для визуализации данных на начальном этапе статистической обработки.
			
			Построение гистограмм используется для получения эмпирической оценки плотности распределения случайной величины. Для построения гистограммы наблюдаемый диапазон изменения случайной величины разбивается на несколько интервалов и подсчитывается доля от всех измерений, попавшая в каждый из интервалов. Величина каждой доли, отнесенная к величине интервала, принимается в качестве оценки значения плотности распределения на соответствующем интервале [\ref{Histogram}].

\section{Реализация}
	Лабораторная работа выполнена на программном языке \emph{Python\;3.8} в среде разработки \emph{Jupyter Notebook\;6.0.3}. В работе использовались следующие пакеты языка \emph{Python}:
	\begin{itemize}
		\item \emph{numpy} --- для генерации выборки и работы с массивами;
		\item \emph{matplotlib.pyplot  и seaborn} --- для построения графиков и гистрограмм;
		\item \emph{scipy.stats} --- содержит все необходимые распределения.
	\end{itemize}
	Ссылка на исходный код лабораторной работы приведена в приложении [\ref{Code}].
\section{Результаты}
	\subsection{Гистограмма и график плотности распределения}
	\begin{center}
		\begin{figure}[h!]
			\includegraphics[width=\textwidth]{"Normal distribution"} 
			\caption[Нормальное распределение]{Нормальное распределение}
		\end{figure}
		
		\begin{figure}[h!]
			\includegraphics[width=\textwidth]{"Cauchy distribution"}
			\caption[Распределение Коши]{Распределение Коши}
		\end{figure}
		
		\begin{figure}[h!]
			\includegraphics[width=\textwidth]{"Laplace distribution"}
			\caption[Распределение Лапласа]{Распределение Лапласа}
		\end{figure}
		
		\begin{figure}[h!]
			\includegraphics[width=\textwidth]{"Poisson distribution"}
			\caption[Распределение Пуассона]{Распределение Пуассона}
		\end{figure}
	
		\begin{figure}[h!]
			\includegraphics[width=\textwidth]{"Uniform distribution"}
			\caption[Равномерное распределение]{Равномерное распределение}
		\end{figure}
	\end{center}

\newpage
\section{Обсуждение}
	Исходя из полученных результатов можно сделать следующие выводы:
	\begin{itemize}
		\item От количества выборки зависит качество оценки плотности распределеня с помощью гистрограммы. Чем больше выборка, тем эта оценка выше.
		\item В гистрограмме могут наблюдатся выбросы, обусловленные вероятностной природой изучаемого процесса, а также разбиением выборки на малое количество интервалов при построении гистрограммы.
	\end{itemize}

\section{Литература}
	\begin{enumerate}
		\item \label{Histogram} Histogram. URL: \url{https://en.wikipedia.org/wiki/Histogram}
	\end{enumerate}
\section{Приложение}
	\begin{enumerate}
		\item \label{Code} Код лабораторной. URL:\url{https://github.com/LuciusGen/Matstat/blob/master/Lab1/Lab1.py}
		
		\item Код отчёта. URL:\url{https://github.com/LuciusGen/Matstat/blob/master/Lab1/Lab1.py}
		
	\end{enumerate}
\end{document}