\documentclass[12pt,a4paper]{article}
\usepackage[utf8]{inputenc}
\usepackage[english,russian]{babel}
\usepackage{indentfirst}
\usepackage{misccorr}


\usepackage{graphicx}%,vmargin}
\usepackage{graphicx}
\DeclareGraphicsExtensions{.png,.jpg}

\usepackage{mathrsfs}
\usepackage{amsmath,amsthm}
\usepackage{hyperref}
\usepackage{float}

\begin{document}
	\begin{titlepage}
		\begin{center}			
			Санкт-Петербургский политехнический университет\\
			Петра Великого
			\vspace{0.25cm}
			
			Институт прикладной математики и механики
			
			Кафедра «Прикладная математика»
			\vfill
			
			\textbf{Отчёт\\
				по лабораторным работам №1-4\\
				по дисциплине\\
				«Математическая статистика»}\\[5mm]
			\bigskip
		\end{center}
		\vfill
		
		\hfill\begin{minipage}{0.45\textwidth}
			Выполнил студент:
			\vspace{0.2cm}
			
			Кондратьев~Д.~А.\\
			группа: 3630102/70301
		\end{minipage}%
		\bigskip
		
		\hfill\begin{minipage}{0.45\textwidth}
			Проверил:
			\vspace{0.2cm}
			
			к.ф.-м.н., доцент\\
			Баженов Александр Николаевич
		\end{minipage}%
		\vfill
		
		\begin{center}
			Санкт-Петербург\\
			2020 г.
		\end{center}
	\end{titlepage}
	
\tableofcontents{}
\newpage
\listoffigures
\listoftables
\newpage

\section{Постановка задачи}
	Для 5-ти рапределений:
	\begin{itemize}
		\item Нормальное распределение $N(x,0,1)$;
		\item Распределение Коши $C(x,0,1)$;
		\item Распределение Лапласа $L( x,0,\frac{1}{\sqrt{2}})$;
		\item Распределение Пуассона $P(k, 10)$;
		\item Равномерное Распределение $U(x,-\sqrt{3}, \sqrt{3})$;
	\end{itemize}

	\begin{enumerate}
		\item Сгенерировать выборки размером 10, 50 и 1000 элементов. Построить на одном рисунке гистограмму и график плотности распределения.
		
		\item Сгенерировать выборки размером 10, 100 и 1000 элементов. Для каждой выборки вычислить следующие статистические характеристики положения данных: $\overline{x}, med\;x, z_R, z_Q, z_{tr}$. Повторить такие вычисления 1000 раз для каждой выборки и найти среднее характеристик положения и их квадратов:
		\begin{equation}\label{eqn:expected_value}
		E(z)=\overline{z}
		\end{equation}
		Вычислить оценку дисперсии по формуле:
		\begin{equation}\label{eqn:variance}
		D(z)=\overline{z^2}-\overline{z}^2
		\end{equation}
		Представить полученные данные в виде таблиц.
		
		\item Сгенерировать выборки размером 20 и 100 элементов. Построить для них боксплот Тьюки. Для каждого распределения определить долю выбросов экспериментально (сгенерировав выборку, соответствующую распределению 1000 раз, и вычислив среднюю долю выбросов и их дисперсии) и сравнить с результатами, полученными теоретически.
		
		\item Сгенерировать выборки размером 20, 60 и 100 элементов. Построить на них эмпирические функции распределения и ядерные оценки плотности распределения на отрезке $[-4;4]$ для непрерывных распределений и на отрезке $[6;14]$ для распределения Пуассона.
	\end{enumerate}
	
\section{Теория}
	\subsection{Распределения}
	\begin{itemize}
		\item Нормальное распределение \begin{equation}\label{eqn:normal}
		N(x,0,1) = \frac{1}{\sqrt{2\pi}}e^{-\frac{x^2}{2}}
		\end{equation}
		
		\item Распределение Коши
		\begin{equation}\label{eqn:cauchy}
		C(x,0,1) = \frac{1}{\pi(1+x^2)}
		\end{equation}
		
		\item Распределение Лапласа
		\begin{equation}\label{eqn:laplace}
		L\left( x,0,\frac{1}{\sqrt{2}}\right) = \frac{1}{\sqrt{2}}e^{-\sqrt{2}\vert x\vert}
		\end{equation}
		
		\item Распределение Пуассона
		\begin{equation}\label{eqn:poisson}
		P(k,10) = \frac{10^k}{k!}e^{-10}
		\end{equation}
		
		\item Равномерное Распределение
		\begin{equation}\label{eqn:uniform}
		U(x,-\sqrt{3}, \sqrt{3}) = 
		\begin{cases}
		\frac{1}{2\sqrt{3}} &\vert x\vert \leqslant \sqrt{3}\\
		0 &\vert x\vert > \sqrt{3}
		\end{cases}
		\end{equation}
	\end{itemize}

	\subsection{Гистограмма}
		\subsubsection{Определение}
			Гистограмма в математической статистике --- это функция, приближающая
			плотность вероятности некоторого распределения, построенная на основе
			выборки из него [\ref{Histogram}].

		\subsubsection{Графическое описание}
			Графически гистограмма строится следующим образом. Сначала множество значений, которое может принимать элемент выборки, разбивается на
			несколько интервалов. Чаще всего эти интервалы берут одинаковыми, но
			это не является строгим требованием. Эти интервалы откладываются на
			горизонтальной оси, затем над каждым рисуется прямоугольник. Если все
			интервалы были одинаковыми, то высота каждого прямоугольника пропорциональна числу элементов выборки, попадающих в соответствующий
			интервал. Если интервалы разные, то высота прямоугольника выбирается
			таким образом, чтобы его площадь была пропорциональна числу элементов
			выборки, которые попали в этот интервал [\ref{Histogram}].

		\subsubsection{Использование}
			Гистограммы применяются в основном для визуализации данных на начальном этапе статистической обработки.
			
			Построение гистограмм используется для получения эмпирической оценки плотности распределения случайной величины. Для построения гистограммы наблюдаемый диапазон изменения случайной величины разбивается на несколько интервалов и подсчитывается доля от всех измерений, попавшая в каждый из интервалов. Величина каждой доли, отнесенная к величине интервала, принимается в качестве оценки значения плотности распределения на соответствующем интервале [\ref{Histogram}].
		
	\subsection{Вариационный ряд}
		Вариационным рядом называется последовательность элементов выборки, расположенных в неубывающем порядке. Одинаковые элементы повторяются  [\ref{Book_1}, с. 409].
		
		Запись вариационного ряда: $x_1, x_2, ... , x_n$.
		
		Элементы вариационного ряда $x_i\ (i = 1,2, ... ,n)$ называются порядковыми статистиками.
	
	\subsection{Выборочные числовые характеристики}
		С помощью выборки образуются её числовые характеристики. Это числовые характеристики дискретной случайной величины $X^*$, принимающей выборочные значения $x_1, x_2, ... , x_n$ [\ref{Book_1}, с. 411].
			
		\subsubsection{Характеристики положения}
			\begin{itemize}
				\item Выборочное среднее:
				\begin{equation}\label{eqn:average}
				\overline{x} = \frac{1}{n}\sum_{i=1}^n x_i
				\end{equation}
				
				\item Выборочная медиана:
				\begin{equation}\label{eqn:med}
				med\;x =
				\begin{cases}
				x_{k+1}, & n = 2k+1\\
				\frac{1}{2}\left(x_k+x_{k+1}\right), & n = 2k
				\end{cases}
				\end{equation}
				
				\item Полусумма экстремальных выборочных элементов:
				\begin{equation}\label{eqn:mean_extr}
				z_R = \frac{1}{2}\left(x_1+x_n\right)
				\end{equation}
				
				\item Полусумма квартилей:
				\begin{equation}\label{eqn:quartiles}
				z_Q = \frac{1}{2}\left(z_{\frac{1}{4}}+z_{\frac{3}{4}}\right) 
				\end{equation}
				
				\item Усечённое среднее:
				\begin{equation}\label{eqn:trim_mean}
				z_{tr} = \frac{1}{n - 2r}\sum_{i=r+1}^{n-r} x_i,\ r\approx\frac{n}{4}
				\end{equation}
			\end{itemize}
		
			\subsubsection{Характеристики рассеяния}
				Выборочная дисперсия:
				\begin{equation}\label{eqn:var}
				D = \frac{1}{n}\sum_{i=1}^n (x_i-\overline{x})^2
				\end{equation}
	\subsection{Боксплот Тьюки}
		\subsubsection{Определение}
			Боксплот (англ. box plot) --- график, использующийся в описательной статистике, компактно изображающий одномерное распределение вероятностей.
		
		\subsubsection{Описание}
			Такой вид диаграммы в удобной форме показывает медиану, нижний и верхний квартили и выбросы. Несколько таких ящиков можно нарисовать бок о бок, чтобы визуально сравнивать одно распределение с другим; их можно располагать как горизонтально, так и вертикально. Расстояния между различными частями ящика позволяют определить степень разброса (дисперсии) и асимметрии данных и выявить выбросы [\ref{Box plot}].
		
		\subsubsection{Построение}
			Границами ящика служат первый и третий квартили, линия в середине
			ящика --- медиана. Концы усов — края статистически значимой выборки
			(без выбросов). Длину «усов» определяют разность первого квартиля и полутора межквартильных расстояний и сумма третьего квартиля и полутора
			межквартильных расстояний. Формула имеет вид:
			\begin{equation}\label{eqn:box_plot}
			X_1 = Q_1 - \frac{3}{2}(Q_3 - Q_1),\  X_2 = Q_3 + \frac{3}{2}(Q_3 - Q_1)
			\end{equation}
			где $X_1$ --- нижняя граница уса, $X_2$ --- верхняя граица уса, $Q_1$ --- первый квартиль, $Q_3$ --- третий квартиль.
			
			Данные, выходящие за границы усов (выбросы), отображаются на графике
			в виде маленьких кружков.
	
	\subsection{Теоретическая вероятность выбросов}
		Встроенными средствами языка программирования \emph{Python} можно вычислить теоретические первый и третий квартили распределений ($Q^T_1$ и $Q^T_3$ соответственно). По формуле (\ref{eqn:box_plot}) можно вычислить теоретические нижнюю и верхнюю границы уса ($X^T_1$ и $X^T_3$ соответственно). Выбросами считаются величины $x$, такие что:
		\begin{equation}\label{eqn:outliers}
		\left[ \begin{array}{l}
		x < X^T_1 \\
		x > X^T_3
		\end{array} \right.
		\end{equation}
		Теоретическая вероятность выбросов для непрерывных распределений:
		\begin{equation}\label{eqn:continous_prob}
		P^T_{out} = P(x < X^T_1) + P(x > X^T_2) = F(X^T_1) + (1-F(X^T_2)),
		\end{equation}
		где $F(X) = P(x \leq X)$ --- функция распределения.
		Теоретическая вероятность выбросов для дискретных распределений:
		\begin{multline}\label{eqn:discrete_prob}
		P^T_{out} = P(x < X^T_1) + P(x > X^T_2)=\\
		= (F(X^T_1) - P(x = X^T_1)) + (1-F(X^T_2)),
		\end{multline}
		где $F(X) = P(x \leq X)$ --- функция распределения.
		
	\subsection{Эмпирическая функция распределения}
		\subsubsection{Статистический ряд}
			Статистическим рядом называется последовательность различных элементов выборки $z_1,z_2, ...,z_k$, расположенных в возрастающем порядке с указанием частот $n_1,n_2, ...,n_k$, с которыми эти элементы содержатся в выборке. Статистический ряд обычно записывается в виде таблицы:
			\begin{table}[h]
				\begin{center}
					\begin{tabular}{|c|c|c|c|c|}
						\hline
						$z$ & $z_1$ & $z_2$ & ... & $z_k$\\
						\hline
						$n$ & $n_1$ & $n_2$ & ... & $n_k$\\
						\hline
					\end{tabular}
				\end{center}
				\caption{Статистический ряд}
			\end{table}
		
		\subsubsection{Определение}
			Эмпирической (выборочной) функцией распределения (э.~ф.~р.) называется относительная частота события $X < x$, полученная по данной выборке:
			\begin{equation}\label{eqn:edf_prob}
			F^*_n(x) = P^*(X < x)
			\end{equation}
		
		\subsubsection{Описание}
			Для получения относительной частоты $P^*(X < x)$ просуммируем в статистическом ряде, построенном по данной выборке, все частоты $n_i$, для которых элементы $z_i$ статистического ряда меньше $x$.\\ Тогда $P^*(X<x) = \frac{1}{n}\sum_{z_i < x}n_i$. Получаем
			\begin{equation}\label{eqn:edf_sum}
			F^*(x) = \frac{1}{n}\sum_{z_i < x}n_i
			\end{equation}
			$F^*(x)$ ---  функция распределения дискретной случайной величины X*, заданной таблицей распределения:
			\begin{table}[h]
				\begin{center}
					\begin{tabular}{|c|c|c|c|c|}
						\hline
						$X^*$ & $z_1$ & $z_2$ & ... & $z_k$\\
						\hline
						$P$ & $\frac{n_1}{n}$ & $\frac{n_2}{n}$ & ... & $\frac{n_k}{n}$\\
						\hline
					\end{tabular}
				\end{center}
				\caption{Таблица распределения}
			\end{table}
			Эмпирическая функция распределения является оценкой, т. е. приближённым значением, генеральной функции распределения:
			\begin{equation}\label{eqn:edf_approx}
			F^*_n(x) \approx F_X(x).
			\end{equation}
		
	\subsection{Оценки плотности вероятности}
		\subsubsection{Определение}
			Оценкой плотности вероятности $f(x)$ называется функция $\hat{f}(x)$, построенная на основе выборки, приближённо равная $f(x)$:
			\begin{equation}\label{eqn:pdf_approx}
			\hat{f}(x) \approx f(x).
			\end{equation}
		
		\subsubsection{Ядерные оценки}
			Представим оценку в виде суммы с числом слагаемых, равным объёму выборки:
			\begin{equation}\label{eqn:pdf_sum}
			\hat{f}(x) = \frac{1}{nh_n}\sum_{i=1}^{n}K(\frac{x-x_i}{h_n}).
			\end{equation}
			Здесь функция $K(u)$, называемая ядерной (ядром), непрерывна и является плотностью вероятности, $x_1, ... ,x_n$ --- элементы выборки, ${h_n}$ --- любая последовательность положительных чисел, обладающая свойствами:
			\begin{equation}\label{eqn:h cond}
			h_n \xrightarrow[n\rightarrow\infty]{}0; \quad \frac{h_n}{n^{-1}}\xrightarrow[n\rightarrow\infty]{}\infty.
			\end{equation}
			Такие оценки называются непрерывными ядерными [\ref{Book_1}, с. 421-423].
			
			Замечание. Свойство, означающее сближение оценки с оцениваемой величиной при $n\rightarrow\infty$ в каком-либо смысле, называется состоятельностью оценки. 
			
			Если плотность $f(x)$ кусочно-непрерывная, то ядерная оценка плотности является состоятельной при соблюдении условий, накладываемых на параметр сглаживания $h_n$, а также на ядро $K(u)$.
			
			Гауссово (нормальное) ядро [\ref{Book_2}, с. 38]:
			\begin{equation}\label{eqn:Kernel fun}
			K(u)=\frac{1}{\sqrt{2\pi}}e^{-\frac{u^2}{2}}.
			\end{equation}
			
			Правило Сильвермана [\ref{Book_2}, с. 44]:
			\begin{equation}\label{eqn:Silverman rule}
			h_n = 1.06\hat{\sigma}n^{-1/5},
			\end{equation}
			где $\hat{\sigma}$ --- выборочное стандартное отклонение.
		
		
		
\section{Реализация}
	Лабораторная работа выполнена на программном языке \emph{Python\;3.8} в среде разработки \emph{Jupyter Notebook\;6.0.3}. В работе использовались следующие пакеты языка \emph{Python}:
	\begin{itemize}
		\item \emph{numpy} --- для генерации выборки и работы с массивами;
		
		\item \emph{matplotlib.pyplot  и seaborn} --- для построения графиков, гистрограмм и боксплотов Тьюки;
		
		\item \emph{scipy.stats} --- содержит все необходимые распределения, а также именно с помощью него можно получить теоретические оценки.
	\end{itemize}
	Ссылка на исходный код лабораторной работы приведена в приложении.
	
\section{Результаты}
	\subsection{Гистограмма и график плотности распределения}
		\begin{center}
		\begin{figure}[H]
			\includegraphics[width=\textwidth]{"../Lab_1/pictures/Normal distribution"} 
			\caption[Нормальное распределение]{Нормальное распределение}
		\end{figure}
		
		\begin{figure}[H]
			\includegraphics[width=\textwidth]{"../Lab_1/pictures/Cauchy distribution"}
			\caption[Распределение Коши]{Распределение Коши}
		\end{figure}
		
		\begin{figure}[H]
			\includegraphics[width=\textwidth]{"../Lab_1/pictures/Laplace distribution"}
			\caption[Распределение Лапласа]{Распределение Лапласа}
		\end{figure}
		
		\begin{figure}[H]
			\includegraphics[width=\textwidth]{"../Lab_1/pictures/Poisson distribution"}
			\caption[Распределение Пуассона]{Распределение Пуассона}
		\end{figure}
	
		\begin{figure}[H]
			\includegraphics[width=\textwidth]{"../Lab_1/pictures/Uniform distribution"}
			\caption[Равномерное распределение]{Равномерное распределение}
		\end{figure}
		\end{center}
	
	\subsection{Характеристики положения и рассеяния}
		\begin{table}[H]
			\begin{center}
				\begin{tabular}{|c|c|c|c|c|c|c|}
					\hline
					& \multicolumn{2}{c|}{$n=10$} & \multicolumn{2}{c|}{$n=100$} & \multicolumn{2}{c|}{$n=1000$}\\
					\hline
					& $E(x)$ (\ref{eqn:expected_value}) & $D(x)$ (\ref{eqn:variance}) & $E(x)$ & $D(x)$ & $E(x)$ & $D(x)$\\
					\hline
					$\overline{x}$ (\ref{eqn:average}) & $0.00$ & $0.0985$ & $0.00$ & $0.0107$ & $-0.002$ & $0.0010$\\
					\hline
					$med\;x$ (\ref{eqn:med}) & $0.0$ & $0.1337$ & $0.00$ & $0.0149$ & $-0.001$ & $0.0016$\\
					\hline
					$z_R$ (\ref{eqn:mean_extr}) & $0.0$ & $0.1847$ & $0.00$ & $0.0928$ & $-0.01$ & $0.0630$\\
					\hline
					$z_Q$ (\ref{eqn:quartiles}) & $0.0$ & $0.1177$ & $-0.01$ & $0.0132$ & $-0.004$ & $0.0012$\\
					\hline
					$z_{tr} (\ref{eqn:trim_mean}) $ & $0.0$ & $0.1113$ & $0.00$ & $0.0123$ & $-0.002$ & $0.0012$\\
					\hline
				\end{tabular}
			\end{center}
			\caption{Нормальное распределение}
		\end{table}
		
		\begin{table}[H]
			\begin{center}
				\begin{tabular}{|c|c|c|c|c|c|c|}
					\hline
					& \multicolumn{2}{c|}{$n=10$} & \multicolumn{2}{c|}{$n=100$} & \multicolumn{2}{c|}{$n=1000$}\\
					\hline
					& $E(x)$ & $D(x)$ & $E(x)$ & $D(x)$ & $E(x)$ & $D(x)$\\
					\hline
					$\overline{x}$ & $2.9$ & $8261.4971$ & $-0.3$ & $282.5623$ & $-0.7$ & $1680.6291$\\
					\hline
					$med\;x$ & $0.00$ & $0.3821$ & $-0.01$ & $0.0260$ & $-0.002$ & $0.0024$\\
					\hline
					$z_R$ & $14.2$ & $206063$ & $-10.5$ & $693687$ & $-358.5$ & $417271337$\\
					\hline
					$z_Q$ & $0.0$ & $1.2177$ & $-0.03$ & $0.0493$ & $-0.003$ & $0.0049$\\
					\hline
					$z_{tr}$ & $0.0$ & $0.5952$ & $0.00$ & $0.0261$ & $-0.001$ & $0.0026$\\
					\hline
				\end{tabular}
			\end{center}
			\caption{Распределение Коши}
		\end{table}
		
		\begin{table}[H]
			\begin{center}
				\begin{tabular}{|c|c|c|c|c|c|c|}
					\hline
					& \multicolumn{2}{c|}{$n=10$} & \multicolumn{2}{c|}{$n=100$} & \multicolumn{2}{c|}{$n=1000$}\\
					\hline
					& $E(x)$ & $D(x)$ & $E(x)$ & $D(x)$ & $E(x)$ & $D(x)$\\
					\hline
					$\overline{x}$ & $-0.02$ & $0.0897$ & $-0.002$ & $0.0099$ & $0.002$ & $0.001$\\
					\hline
					$med\;x$ & $-0.02$ & $0.0648$ & $0.003$ & $0.0058$ & $0.0003$ & $0.0005$\\
					\hline
					$z_R$ & $0.0$ & $0.3887$ & $0.0$ & $0.409$ & $0.0$ & $0.4084$\\
					\hline
					$z_Q$ & $-0.01$ & $0.0943$ & $-0.01$ & $0.01$ & $-0.001$ & $0.001$\\
					\hline
					$z_{tr}$ & $-0.01$ & $0.0662$ & $0.001$ & $0.0063$ & $0.0007$ & $0.0006$\\
					\hline
				\end{tabular}
			\end{center}
			\caption{Распределение Лапласа}
		\end{table}
		
		\begin{table}[H]
			\begin{center}
				\begin{tabular}{|c|c|c|c|c|c|c|}
					\hline
					& \multicolumn{2}{c|}{$n=10$} & \multicolumn{2}{c|}{$n=100$} & \multicolumn{2}{c|}{$n=1000$}\\
					\hline
					& $E(x)$ & $D(x)$ & $E(x)$ & $D(x)$ & $E(x)$ & $D(x)$\\
					\hline
					$\overline{x}$ & $10.02$ & $1.0132$ & $10.00$ & $0.0976$ & $9.994$ & $0.0098$\\
					\hline
					$med\;x$ & $9.9$ & $1.4617$ & $9.9$ & $0.1944$ & $9.995$ & $0.005$\\
					\hline
					$z_R$ & $10.4$ & $1.9290$ & $10.95$ & $1.0299$ & $11.8$ & $0.6673$\\
					\hline
					$z_Q$ & $9.9$ & $1.2086$ & $9.9$ & $0.1590$ & $9.995$ & $0.0037$\\
					\hline
					$z_{tr}$ & $9.9$ & $1.1459$ & $9.9$ & $0.1166$ & $9.85$ & $0.0107$\\
					\hline
				\end{tabular}
			\end{center}
			\caption{Распределение Пуассона}
		\end{table}
		
		\newpage
		\begin{table}[H]
			\begin{center}
				\begin{tabular}{|c|c|c|c|c|c|c|}
					\hline
					& \multicolumn{2}{c|}{$n=10$} & \multicolumn{2}{c|}{$n=100$} & \multicolumn{2}{c|}{$n=1000$}\\
					\hline
					& $E(x)$ & $D(x)$ & $E(x)$ & $D(x)$ & $E(x)$ & $D(x)$\\
					\hline
					$\overline{x}$ & $-0.02$ & $0.0893$ & $0.000$ & $0.0099$ & $0.001$ & $0.001$\\
					\hline
					$med\;x$ & $0.0$ & $0.2088$ & $0.01$ & $0.0294$ & $0.001$ & $0.003$\\
					\hline
					$z_R$ & $-0.01$ & $0.0392$ & $0.0002$ & $0.0006$ & $0.0001$ & $0.0000$\\
					\hline
					$z_Q$ & $0.0$ & $0.1282$ & $-0.02$ & $0.0146$ & $-0.001$ & $0.0014$\\
					\hline
					$z_{tr}$ & $0.0$ & $0.1479$ & $0.00$ & $0.0200$ & $0.001$ & $0.0019$\\
					\hline
				\end{tabular}
			\end{center}
			\caption{Равномерное распределение}
		\end{table}

	\subsection{Боксплот Тьюки}
	\begin{center}
		\begin{figure}[H]
			\includegraphics[width=\textwidth]{"../Lab_3/pictures/Normal distribution"} 
			\caption[Нормальное распределение]{Нормальное распределение}
		\end{figure}
	
		\begin{figure}[H]
			\includegraphics[width=\textwidth]{"../Lab_3/pictures/Cauchy distribution"}
			\caption[Распределение Коши]{Распределение Коши}
			
			\includegraphics[width=\textwidth]{"../Lab_3/pictures/Laplace distribution"}
			\caption[Распределение Лапласа]{Распределение Лапласа}
		\end{figure}
		
		\begin{figure}[H]
			\includegraphics[width=\textwidth]{"../Lab_3/pictures/Poisson distribution"}
			\caption[Распределение Пуассона]{Распределение Пуассона}
			
			\includegraphics[width=\textwidth]{"../Lab_3/pictures/Uniform distribution"}
			\caption[Равномерное распределение]{Равномерное распределение}
		\end{figure}
	\end{center}

	\subsection{Доля выбросов}
	\begin{table}[H]
		\begin{center}
			\begin{tabular}{|c|c|c|c|c|}
				\hline
				& \multicolumn{2}{c|}{$n=20$} & \multicolumn{2}{c|}{$n=100$}\\
				\cline{2-5}
				\raisebox{1.5ex}[0cm][0cm]{Распределение}
				& $E$ (\ref{eqn:expected_value}) & $D$ (\ref{eqn:variance}) & $E$ & $D$\\
				\hline
				Нормальное распределение & $0.024$ & $0.0019$ & $0.0101$ & $0.0002$ \\
				\hline
				Распределение Коши & $0.152$ & $0.0049$ & $0.154$ & $0.0011$\\
				\hline
				Распределение Лапласа & $0.074$ & $0.0044$ & $0.0632$ & $0.0009$\\
				\hline
				Распределение Пуассона & $0.024$ & $0.0022$ & $0.0105$ & $0.0002$\\
				\hline
				Равномерное распределение & $0.0020$ & $0.0002$ & $0.0$ & $0.0$\\
				\hline
			\end{tabular}
		\end{center}
		\caption{Доля выбросов}
	\end{table}

	\subsection{Теоретическая вероятность выбросов}
		\begin{table}[H]
			\begin{center}
				\begin{tabular}{|c|c|c|c|c|c|}
					\hline
					Распределение & $Q^T_1$ & $Q^T_3$ & $X^T_1$ (\ref{eqn:box_plot}) & $X^T_2$ (\ref{eqn:box_plot}) & $P^T_{out}$ (\ref{eqn:continous_prob}), (\ref{eqn:discrete_prob})\\
					\hline
					Нормальное распределение & $-0.6745$ & $0.6745$ & $-2.698$ & $2.698$ & $0.007$\\
					\hline
					Распределение Коши & $-1$ & $1$ & $-4$ & $4$ & $0.156$\\
					\hline
					Распределение Лапласа & $-0.4901$ & $0.4901$ & $-1.9605$ & $1.9605$ & $0.0625$\\
					\hline
					Распределение Пуассона & $8$ & $12$ & $2$ & $18$ & $0.0077$\\
					\hline
					Равномерное распределение & $-0.866$ & $0.866$ & $-3.4641$ & $3.4641$ & $0$\\
					\hline
				\end{tabular}
			\end{center}
			\caption{Теоретическая вероятность выбросов}
		\end{table}
	
	\subsection{Эмпирическая функция распределения}
		\begin{center}
			\begin{figure}[H]
				\includegraphics[width=\textwidth]{"../Lab_4/pictures/Normal distribution (edf)"} 
				\caption[Нормальное распределение]{Нормальное распределение}
			\end{figure}
			
			\begin{figure}[H]
				\includegraphics[width=\textwidth]{"../Lab_4/pictures/Cauchy distribution (edf)"}
				\caption[Распределение Коши]{Распределение Коши}
			\end{figure}
			
			\begin{figure}[H]
				\includegraphics[width=\textwidth]{"../Lab_4/pictures/Laplace distribution (edf)"}
				\caption[Распределение Лапласа]{Распределение Лапласа}
			\end{figure}
			
			\begin{figure}[H]
				\includegraphics[width=\textwidth]{"../Lab_4/pictures/Poisson distribution (edf)"}
				\caption[Распределение Пуассона]{Распределение Пуассона}
			\end{figure}
			
			\begin{figure}[H]			
				\includegraphics[width=\textwidth]{"../Lab_4/pictures/Uniform distribution (edf)"}
				\caption[Равномерное распределение]{Равномерное распределение}
			\end{figure}
		\end{center}
	
	\subsection{Ядерные оценки плотности распределения}
		\begin{center}
			\begin{figure}[H]
				\includegraphics[width=\textwidth]{"../Lab_4/pictures/Normal distribution (kernel)"} 
				\caption[Нормальное распределение]{Нормальное распределение}
			\end{figure}
			
			\begin{figure}[H]
				\includegraphics[width=\textwidth]{"../Lab_4/pictures/Cauchy distribution (kernel)"}
				\caption[Распределение Коши]{Распределение Коши}
			\end{figure}
			
			\begin{figure}[H]
				\includegraphics[width=\textwidth]{"../Lab_4/pictures/Laplace distribution (kernel)"}
				\caption[Распределение Лапласа]{Распределение Лапласа}
			\end{figure}
			
			\begin{figure}[hp!]
				\includegraphics[width=\textwidth]{"../Lab_4/pictures/Poisson distribution (kernel)"}
				\caption[Распределение Пуассона]{Распределение Пуассона}
			\end{figure}
			
			\begin{figure}[hp!]			
				\includegraphics[width=\textwidth]{"../Lab_4/pictures/Uniform distribution (kernel)"}
				\caption[Равномерное распределение]{Равномерное распределение}
			\end{figure}
		\end{center}
	
\newpage
\section{Обсуждение}
	\subsection{Гистограмма и график плотности распределения}
		Исходя из полученных результатов можно сделать следующие выводы:
		\begin{itemize}
			\item От количества выборки зависит качество оценки плотности распределеня с помощью гистрограммы. Чем больше выборка, тем эта оценка выше.
			
			\item В гистрограмме могут наблюдатся выбросы, обусловленные вероятностной природой изучаемого процесса, а также разбиением выборки на малое количество интервалов при построении гистрограммы.
		\end{itemize}

	\subsection{Характеристики положения и рассеяния}
		Исходя из полученных результатов можно сделать следующие выводы:
		\begin{itemize}
			\item Дисперсия может гарантировать порядок точности среднего значения только до первой значащей цифры после запятой в дисперсии включительно. Поэтому есть необходимость проверки данных результатов на совпадение этой точности.
			
			\item При увеличении количества выборки гарантируемая точность будет возрастать.
			
			\item Предыдущие выводы касаются всех распределений, кроме распределения Коши. Так как оно имеет бесконечную дисперсию и, следовательно, никакой точности гарантировать не может.
			
			\item Можно вывести следующее отношение для характеристик положения при $n=1000$:
			\begin{enumerate}
				\item для нормального распределения: $z_R < z_Q < z_{tr} \leq \overline{x} < med\;x$;
				
				\item для распределения Коши: $z_R < \overline{x} < z_Q < med\;x < z_{tr}$;
				
				\item для распределения Лапласа: $z_Q < z_R < med\;x < z_{tr} < \overline{x}$;
				
				\item для распределения Пуассона: $ z_{tr} < \overline{x} < z_Q \leq med\;x < z_R$;
				
				\item для равномерного распределения: $z_Q < z_R < z_{tr} \leq med\;x \leq \overline{x}$.
			\end{enumerate}
		\end{itemize}
	
	\subsection{Боксплот Тьюки и доля выбросов}
		Исходя из полученных результатов можно сделать следующие выводы:
		\begin{itemize}
			\item Справедливо следующее соотношение для долей выбросов:\\
			равномерное распределение < нормальное распределение < распределение Пуассона < распределение Лапласа < распределение Коши.
			
			\item Доля выбросов, полученная экспериментально, близка с результатами, полученными теоретически.
		\end{itemize}
	
	\subsection{Эмпирическая функция и ядерные оценки плотности распределения}
		Исходя из полученных результатов можно сделать следующие выводы:
		\begin{itemize}
			\item При увеличении размера выборки  качество оценки эмперической функцией эталонной функции распределения возрастает.
			
			\item При увеличении размера выборки  качество ядерной оценки плотности распределения возрастает.
			
			\item Наилучшее качество ядерной оценки плотности распределения достигается при следующих параметрах:
			\begin{enumerate}
				\item для нормального распределения --- $h=h_n$;
				
				\item для распределения Коши --- $h=h_n$;
				
				\item для распределения Лапласа --- $h=\frac{h_n}{2}$ или $h=h_n$;
				
				\item для распределения Пуассона --- $h=2h_n$;
				
				\item для равномерного распределения --- $h=\frac{h_n}{2}$ или $h=h_n$.
			\end{enumerate}
		\end{itemize}



\section{Литература}
	\begin{enumerate}
		\item \label{Book_1} \textbf{Вероятностные разделы математики.} Учебник для бакалавров технических направлений.//Под ред. Максимова~Ю.Д. --- Спб.: «Иван Федоров», 2001. --- 592 c., илл.
		
		\item \label{Book_2} Анатольев, Станислав (2009) «Непараметрическая регрессия», Квантиль, №7, стр. 37-52.
		
		\item \label{Histogram} Histogram. URL: \url{https://en.wikipedia.org/wiki/Histogram};
		
		\item \label{Box plot} Box plot. URL: \url{https://en.wikipedia.org/wiki/Box_plot};
		
		\item Kernel density estimation. URL: \url{https://en.wikipedia.org/wiki/Kernel_density_estimation}.
	\end{enumerate}

\section{Приложение}
	\begin{enumerate}
		\item Код лабораторной №1. URL: \url{https://github.com/DmitriiKondratev/MatStat/blob/master/Lab_1/Lab_1.ipynb}
		
		\item Код отчёта №1. URL: \url{https://github.com/DmitriiKondratev/MatStat/blob/master/Lab_1/Lab_report_1.tex}
		
		\item Код лабораторной №2. URL: \url{https://github.com/DmitriiKondratev/MatStat/blob/master/Lab_2/Lab_2.ipynb}
		
		\item Код отчёта №2. URL: \url{https://github.com/DmitriiKondratev/MatStat/blob/master/Lab_2/Lab_report_2.tex}
		
		\item Код лабораторной №3. URL: \url{https://github.com/DmitriiKondratev/MatStat/blob/master/Lab_3/Lab_3.ipynb}
		
		\item Код отчёта №3. URL: \url{https://github.com/DmitriiKondratev/MatStat/blob/master/Lab_3/Lab_report_3.tex}
		
		\item Код лабораторной №4. URL: \url{https://github.com/DmitriiKondratev/MatStat/blob/master/Lab_4/Lab_4.ipynb}
		
		\item Код отчёта №4. URL: \url{https://github.com/DmitriiKondratev/MatStat/blob/master/Lab_4/Lab_report_4.tex}
		
		\item Код общего отчёта №1-4. URL: \url{https://github.com/DmitriiKondratev/MatStat/blob/master/Lab_1-4/Lab_report_1-4.tex}	
	\end{enumerate}
\end{document}